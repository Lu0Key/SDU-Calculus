% !TeX program = XeLaTeX
% !TeX root = main.tex
\chapter{导数与微分}\label{cha:2}

% \begin{enumerate}\setlength{\itemsep}{7pt}
% \end{enumerate}

% \begin{enumerate}[(1)]\setlength{\itemsep}{5pt}\setlength{\topsep}{15pt}
% \end{enumerate}

\section{导数的概念}

\begin{enumerate}\setlength{\itemsep}{7pt}
    \item 将一个物体垂直上抛,设经过时间 $t$ 秒后,物体上升的高度为 $s=10t-\dfrac{1}{2}gt^2$. 
    求下列各值:
    \begin{enumerate}[(1)]\setlength{\itemsep}{5pt}\setlength{\topsep}{15pt}
        \item 物体在 $1$ 秒到 $(1+\Delta t)$ 秒这段时间内的平均速度;
        \item 物体在 $1$ 秒时的瞬时速度;
        \item 物体在 $t_0$ 秒到 $(t_0+\Delta t)$ 秒这段时间内的平均速度;
        \item 物体在 $t_0$ 秒的瞬时速度.
    \end{enumerate}

    \item 利用导数定义求下列函数在 $x=0$ 处的导数 $f'(0)$:
    \begin{enumerate}[(1)]\setlength{\itemsep}{5pt}\setlength{\topsep}{15pt}
        \item $f(x)=x^2+x+1$;
        \item $f(x)=\cos(x+3)$;
        \item $f(x)=x(x+1)(x+2)\cdots(x+n)$;
        \item $f(x)=\begin{cases}
            \dfrac{x}{1+e^{\frac{1}{x}}},&x<0,\\
            0,&x=0,\\
            \dfrac{2x}{1+e^x},&x>0.
        \end{cases}$
    \end{enumerate}

    \item 设 $f(x)$ 在点 $x_0$ 处可导,按照导数定义,指出 $A$ 表示什么?
    \begin{enumerate}[(1)]\setlength{\itemsep}{5pt}\setlength{\topsep}{15pt}
        \item $\displaystyle\lim_{\Delta x\to0}\dfrac{f(x_0-\Delta x)-f(x_0)}{\Delta x}=A$;
        \item $\displaystyle\lim_{h\to0}\dfrac{f(x_0)-f(x_0-h)}{h}=A$;
        \item $\displaystyle\lim_{h\to0}\dfrac{f(x_0+h)-f(x_0)}{h}=A$;
        \item $\displaystyle\lim_{h\to0}\dfrac{f(x_0+3h)-f(x_0)}{h}=A$;
        \item $\displaystyle\lim_{h\to0}\dfrac{f(x_0+h^2)-f(x_0)}{h^2}=A$;
        \item $\displaystyle\lim_{x\to0}\dfrac{f(x)}{x}=A$,设 $f(0)=0$,且 $f'(0)$ 存在.
    \end{enumerate}

    \item 讨论下列函数在指定点的连续性与可导性:
    \begin{enumerate}[(1)]\setlength{\itemsep}{5pt}\setlength{\topsep}{15pt}
        \item $f(x)=\begin{cases}
            \sin x,&x\geqslant0,\\
            x,&x<0,
        \end{cases}$ 在 $x=0$;
        \item $f(x)=|x|x$,在 $x=0$.
    \end{enumerate}

    \item 设 $f(x)$ 为可导函数,且满足条件 $\displaystyle\lim_{x\to0}\dfrac{f(1)-f(1-x)}{2x}=-1$,
    求曲线 $y=f(x)$ 在点 $(1,f(1))$ 处的切线斜率.

    \item 已知 $f'(x_0)=-1$,求 $\displaystyle\lim_{x\to0}\dfrac{x}{f(x_0-2x)-f(x_0-x)}$.
    
    \item 已知 $f(x)$ 满足条件 $f(1+x)=af(x)$,且 $f'(0)=b$(常数 $a,\;b\not=0$),求 $f'(1)$.
    
    \item 某城镇每户居民用水费用(元)的模型由下式确定:
    \[
        f(x)=\begin{cases}
            1.75x,&0\leqslant x\leqslant 4,\\
            7+3.2(x-4),&x>4,
        \end{cases}
    \]
    试问:\begin{minipage}[t]{10cm}
            \begin{enumerate}[(1)]
            \item 函数 $f(x)$ 在 $x=4$ 处是否连续;
            \item 函数 $f(x)$ 在 $x=4$ 处是否可导.
            \end{enumerate}
        \end{minipage}
\end{enumerate}

\section{导数的基本公式与运算法则}

\begin{enumerate}\setlength{\itemsep}{7pt}
    \item 求下列函数的导数:
    \begin{enumerate}[(1)]\setlength{\itemsep}{5pt}\setlength{\topsep}{15pt}
        \item $y=x^3-4\sqrt{x}+\dfrac{1}{x^2}+12$;
        \item $y=\ln x+3\log_2x$;
        \item $y=\sin x\cdot \cos x$;
        \item $y=\dfrac{\sin x}{x^2}$;
        \item $y=\dfrac{x-1}{x+1}$;
        \item $y=\dfrac{2\csc x}{1+x^2}$.
    \end{enumerate}

    \item 求系列函数在给定点处的导数:
    \begin{enumerate}[(1)]\setlength{\itemsep}{5pt}\setlength{\topsep}{15pt}
        \item $y=\sin x-\cos x$, 求 $y'\big|_{x=\frac{\pi}{6}}$ 和 $y'\big|_{x=\frac{\pi}{4}}$;
        \item $y=(x+e^{-\frac{\pi}{2}})^{\frac{2}{3}}$,求 $y'\big|_{x=0}$;
        \item $y=\arctan e^{x}-\ln\sqrt{\dfrac{e^{2x}}{e^{2x}+1}}$,
        求 $\dfrac{\text{d}y}{\text{d}x}\bigg|_{x=1}$;
        \item $f(x)=\dfrac{3}{5-x}+\dfrac{x^2}{5}$,求 $f'(0)$ 和 $f'(2)$.
    \end{enumerate}

    \item 求下列函数的导数:
    \begin{enumerate}[(1)]\setlength{\itemsep}{5pt}\setlength{\topsep}{15pt}
        \item $y=(3x^2-2x+1)^4$;
        \item $y=\cos(\sin x)$;
        \item $y=\ln(1+x^2)$;
        \item $y=\log_a(\sin^2x+2)$;
        \item $y=\ln\arctan\dfrac{1}{x}$;
        \item $y=\left(\arctan\dfrac{x}{2}\right)^3$;
        \item $y=\ln[\ln(\ln x)]$;
        \item $y=e^{\tan\frac{1}{x}}\cdot\sin\dfrac{1}{x}$.
    \end{enumerate}

    \item 已知 $y=f\left(\dfrac{3x-2}{3x+2}\right)$,$f'(x)=\arctan x^2$,
    求 $\dfrac{\text{d}y}{\text{d}x}\bigg|_{x=0}$.
    
    \item 已知 $h(x)=e^{1+g(x)},\;h'(1)=1.\;g'(1)=2$,求 $g(1)$.
    
    \item 设可导函数 $f(x)$ 满足 $af(x)+bf\left(\dfrac{1}{x}\right)=\dfrac{c}{x}$,
    式中 $a,\;b,\;c$ 为常数,且 $|a|\not=|b|$,求 $f'(x)$.

    \item[*7.] 设函数 $f(x)$ 在 $x=0$ 的领域内连续,
    且 $\displaystyle\lim_{x\to0}\dfrac{f(x)}{\sqrt{1+x}-1}=-1$,求 $f'(0)$. 

    \item[8.] 设 $f(x)=\begin{cases}
        x\arctan\dfrac{1}{x^2},&x\not=0,\\
        0,&x=0,
    \end{cases}$ 试谈论 $f'(x)$ 在 $x=0$ 处的连续性.

    \item[9.] 抛物线 $y=x^2-5x+6$ 上哪一点的法线平行于直线 $x+y-1=0$,并求出此法线的方程.
    
    \item[10.] 设某产品一周的产量为 $Q(x)=200x+6x^2$,其中 $x$ 装配线上劳动者的人数. 如果现在有60人在装配线上,
    \begin{enumerate}[(1)]\setlength{\itemsep}{5pt}\setlength{\topsep}{15pt}
        \item 计算 $Q(61)-Q(60)$,看看一周产量的实际变化;
        \item 求 $Q'(60)$,并解释:由于增加一个人而导致的一周产量的变化.
    \end{enumerate}

\end{enumerate}

