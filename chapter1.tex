% !TeX program = XeLaTeX
% !TeX root = main.tex
\chapter{函数、极限和连续}\label{cha:11}

% \begin{enumerate}\setlength{\itemsep}{7pt}
% \end{enumerate}

% \begin{enumerate}[(1)]\setlength{\itemsep}{5pt}\setlength{\topsep}{15pt}
% \end{enumerate}

\section{函数}

\begin{enumerate}\setlength{\itemsep}{7pt}
    \item 求下列函数的定义域:
    \begin{enumerate}[(1)]\setlength{\itemsep}{5pt}\setlength{\topsep}{15pt}
        \item $y=\sin\sqrt{x-1}$;
        \item $y=\lg(x^2-4)$;
        \item $y=\arccos\dfrac{x-3}{2}$;
        \item $y=\dfrac{1}{\sin x-\cos x}$;
        \item $y=\sqrt{1-2x}+\sqrt{e-e\left(\dfrac{3x-1}{2}\right)^2}$;
        \item $y=\dfrac{1}{1+\dfrac{1}{1+\dfrac{1}{x}}}$.
    \end{enumerate}

    \item 判断下列各题中两个函数是否相同,为什么?
    \begin{enumerate}[(1)]\setlength{\itemsep}{5pt}\setlength{\topsep}{15pt}
        \item $y=\dfrac{x^2}{|x|}$ 和 $y=x$;
        \item $y=|x|$ 和 $y=\sqrt{x^2}$;
        \item $y=\log_ax^2$ 和 $y=2\log_ax(a>0,\;a\not=1)$;
        \item $y=\dfrac{x^2-4}{x+2}$ 和 $y=x-2$.
    \end{enumerate}

    \item 下列函数中哪些是偶函数,哪些是奇函数,哪些是非奇非偶函数?
    \begin{enumerate}[(1)]\setlength{\itemsep}{5pt}\setlength{\topsep}{15pt}
        \item $y=x^2\sin x$;
        \item $y=x^2+\sin x$;
        \item $y=x(x-1)(x+1)$;
        \item $y=\dfrac{a^x+a^{-x}}{2}$;
        \item $y=\ln\dfrac{x+1}{x-1}$;
        \item $y=|x+1|$.
    \end{enumerate}

    \item 设
    \[
        \varphi(x)=\begin{cases}
            |\sin x|,&|x|<\dfrac{\pi}{3},\\
            0,&|x|\geqslant\dfrac{\pi}{3},
        \end{cases}
    \]
    求 $\varphi\left(\dfrac{\pi}{6}\right),\;\varphi\left(\dfrac{\pi}{4}\right),\;\varphi\left(-\dfrac{\pi}{4}\right),\;\varphi\left(-2\right)$,
    并作出函数 $y=\varphi(x)$ 的图形.

    \item 指出下列函数在指定区间内的单调性:
    \begin{enumerate}[(1)]\setlength{\itemsep}{5pt}\setlength{\topsep}{15pt}
        \item $y=|x+1|,\quad x\in[-5,\;-1]$;
        \item $y=a^x(a>0,\;a\not=1),\quad x\in(-\infty,+\infty)$.
    \end{enumerate}

    \item 指出下列函数中哪些是周期函数,并指出其周期:
    \begin{enumerate}[(1)]\setlength{\itemsep}{5pt}\setlength{\topsep}{15pt}
        \item $y=\cos 4x$;
        \item $y=x\cos x$;
        \item $y=1+\sin \pi x$;
        \item $y=\sin^2x$.
    \end{enumerate}

    \item 求下列函数的反函数:
    \begin{enumerate}[(1)]\setlength{\itemsep}{5pt}\setlength{\topsep}{15pt}
        \item $y=2\sin 3x\quad\left(0<x<\dfrac{\pi}{6}\right)$;
        \item $y=\sqrt[3]{x+1}$;
        \item $y=\dfrac{ax+b}{cx+d}\;\;(ad-bc\not=0)$;
        \item $y=\dfrac{1+\sqrt{1-x}}{1-\sqrt{1-x}}$.
    \end{enumerate}

    \item 函数 $y=\sin\dfrac{\pi x}{2(1+x^2)}$ 的值域是?
    
    \item 求初等函数 $f(x)$ 的表达式:
    \begin{enumerate}[(1)]\setlength{\itemsep}{5pt}\setlength{\topsep}{15pt}
        \item $f\left(x+\dfrac{1}{x}\right)=\dfrac{x^2}{x^4+1}$;
        \item $f\left(\dfrac{x+1}{x-1}\right)=3f(x)-2x$.
    \end{enumerate}

    \item 设 $f(x)=\begin{cases}
        1,&|x|\leqslant 1,\\
        0,&|x|>1,
    \end{cases}$ 则 $f\{f[f(x)]\}$ 等于?

    \item 设 $f(x)=x^2,\;g(x)=2^x$,求 $f[f(x)],\;f[g(x)],\;g[f(x)],\;g[g(x)]$.
    
    \item 将下列各点的极坐标化为直角坐标:\\
    $A\left(-3,\;\dfrac{\pi}{2}\right),\quad B(2,\;-\pi),\quad C(-3,\;3),\quad D(-5,\;0)$.

    \item 将下列 个点的直角坐标转化为极坐标:\\
    $A(-3,\;-4),\quad B(3,\;4),\quad C(-2,\;\pi),\quad D(\pi,\;3)$.

    \item 化下列曲线的极坐标方程为直角坐标方程:
    \begin{enumerate}[(1)]\setlength{\itemsep}{5pt}\setlength{\topsep}{15pt}
        \item $\theta=\dfrac{\pi}{4}$;
        \item $r=2a\cos\theta$;
        \item $r=\dfrac{1}{12\cos \theta}$;
        \item $r=\dfrac{1}{2\sin\theta-3\cos\theta}$.
    \end{enumerate}    

    \item 设我国少年儿童的平均身高 $y(cm)$ 与年龄 $x$ 的函数表达式为 $y=ax+b$,已知 $1$ 岁儿童平均身高为 $85\;cm$,$10$ 岁儿童平均身高为 $130cm$,求 $a,\;b$.


\end{enumerate}


\section{极限}

\begin{enumerate}\setlength{\itemsep}{7pt}
    \item 观察数列 $\{x_n\}$ 的变化趋势,讨论它们的极限是否存在:
    \begin{enumerate}[(1)]\setlength{\itemsep}{5pt}\setlength{\topsep}{15pt}
        \item $x_n=\dfrac{1}{2^n}$;
        \item $x_n=(-1)^n\dfrac{1}{n^2}$;
        \item $x_n=(-1)^nn$;
        \item $x_n=\dfrac{n-1}{n+1}$;
        \item $x_n=n-\dfrac{1}{n}$.
    \end{enumerate}

    \item 讨论下列各函数在点 $x=0$ 处的极限是否存在:
    \begin{enumerate}[(1)]\setlength{\itemsep}{5pt}\setlength{\topsep}{15pt}
        \item $f(x)=\begin{cases}
            2,&x=0,\\
            1&x\not=0;
        \end{cases}$
        \item $f(x)=\begin{cases}
            x+1,&x<0,\\
            0,&x=0,\\
            1-x,x>0;
        \end{cases}$
        \item $f(x)=\begin{cases}
            x+1,&x\leqslant0,\\
            x,x>0;
        \end{cases}$
        \item $f(x)=\begin{cases}
            \dfrac{1}{x},&x\not=0,\\
            0,&x=0;
        \end{cases}$
        \item $f(x)=\begin{cases}
            \dfrac{1}{\sin x},&0<|x|\leqslant\dfrac{\pi}{2},\\
            1,&x=0.
        \end{cases}$
    \end{enumerate}

    \item 设 $f(x)=\begin{cases}
        \sin x,&x>0,\\
        a+x^2,&x<0,
    \end{cases}$ 问 $a$ 为何值时,$\displaystyle\lim_{x\to0}f(x)$ 存在?极限值为多少?

    \item 设数列 $\{x_n\}$ 的一般项 $x_n=\dfrac{1}{n}\cos\dfrac{n\pi}{2}$. 求 $\displaystyle\lim_{n\to\infty}x_n$. 
    存在 $N$,使当 $n>N$ 时,$x_n$ 与其极限之差的绝对值小于正数 $\varepsilon$,求出 $N$. 当 $\varepsilon=0.001$ 时,再求出数 $N$.

    \item 试求出下列各极限:
    \begin{enumerate}[(1)]\setlength{\itemsep}{5pt}\setlength{\topsep}{15pt}
        \item $\displaystyle\lim_{n\to\infty}\dfrac{3n^2-2}{8n^2+n}$;
        \item $\displaystyle\lim_{n\to\infty}(\sqrt{n+1}-\sqrt{n})$;
        \item $\displaystyle\lim_{x\to0}\dfrac{x^2-2}{4x^2+x+2}$;
        \item $\displaystyle\lim_{x\to2}\dfrac{x^2-2}{4x^2+x+3}$;
        \item $\displaystyle\lim_{x\to\infty}\dfrac{x^2-2}{4x^2+x+3}$;
        \item $\displaystyle\lim_{x\to+\infty}\dfrac{\sqrt{x+1}-\sqrt{x-1}}{x}$;
        \item $\displaystyle\lim_{n\to\infty}\dfrac{1+\dfrac{1}{2}+\dfrac{1}{4}+\cdots+\dfrac{1}{2^n}}{1+\dfrac{1}{3}+\dfrac{1}{9}+\cdots+\dfrac{1}{3^n}}$;
        \item $\displaystyle\lim_{x\to1}\dfrac{x-1}{\sqrt{x}-1}$
        \item $\displaystyle\lim_{x\to\infty}\left(2-\dfrac{1}{x^2}\right)\left(5+\dfrac{1}{x}\right)$;
        \item $\displaystyle\lim_{x\to1}\left(\dfrac{1}{1-x}-\dfrac{3}{1-x^3}\right)$.
    \end{enumerate}

    \item 设 $f(x)=\begin{cases}
        e^{\frac{1}{x}}+1,&x<0,\\
        1,&x=0,\\
        1+x\sin \dfrac{1}{x},&x>0,
    \end{cases}$
    求 $\displaystyle\lim_{x\to0}f(x)$.

    \item 设函数 $f(x)=a^x(a>0,\;a\not1)$,则 $\displaystyle\lim_{x\to\infty}\dfrac{1}{n^2}\ln[f(1)f(2)\cdots f(n)]=$?
    
    \item 已知 $\displaystyle\lim_{n\to\infty}\left(\dfrac{x^2}{x+1}-ax-b\right)=0$,求常数 $a,\;b$.
    
    \item  求 $\displaystyle\lim_{x\to0}\dfrac{\cos x+\cos^2x+\cdots+\cos^n x-n}{\cos x-1}$.
    
    \item 根据定义证明:
    \begin{enumerate}[(1)]\setlength{\itemsep}{5pt}\setlength{\topsep}{15pt}
        \item 当 $x \to 3$ 时 $y=\dfrac{x^2-9}{x+3}$ 为无穷小;
        \item 当 $x \to 0$ 时 $y=x\sin\dfrac{1}{x}$ 为无穷小.
    \end{enumerate}
    
    \item 函数 $y=x\cos x$ 在 $(-\infty, +\infty)$ 内是否有界?当 $x\to\infty$ 时这个函数是否为无穷大?为什么?
    
    \item 下列陈述中哪些是对的,哪些是错的?如果是对的,说明理由;如果是错的,试给出一个反例.
    \begin{enumerate}[(1)]\setlength{\itemsep}{5pt}\setlength{\topsep}{15pt}
        \item 如果 $\displaystyle\lim_{x\to x_0}f(x)$ 存在,但 $\displaystyle\lim_{x\to x_0}g(x)$ 不存在,那么 $\displaystyle\lim_{x\to x_0}[f(x)+g(x)]$ 不存在;
        \item 如果 $\displaystyle\lim_{x\to x_0}f(x)$ 和 $\displaystyle\lim_{x\to x_0}g(x)$ 都不存在,那么 $\displaystyle\lim_{x\to x_0}[f(x)+g(x)]$ 不存在;
        \item 如果 $\displaystyle\lim_{x\to x_0}f(x)$ 存在,但 $\displaystyle\lim_{x\to x_0}g(x)$ 不存在,那么 $\displaystyle\lim_{x\to x_0}[f(x) \cdot g(x)]$ 不存在.
    \end{enumerate}



\end{enumerate}

\section{极限存在准则及两个重要极限}


\section{连续}