% !TeX program = XeLaTeX
% !TeX root = main.tex
\chapter{常微分方程及差分方程}\label{cha:5}

% \begin{enumerate}\setlength{\itemsep}{7pt}
% \end{enumerate}

% \begin{enumerate}[(1)]\setlength{\itemsep}{5pt}\setlength{\topsep}{15pt}
% \end{enumerate}

\section{微分方程的基本概念}

\begin{enumerate}\setlength{\itemsep}{7pt}
    \item 指出下列微分方程的阶数:
    \begin{enumerate}[(1)]\setlength{\itemsep}{5pt}\setlength{\topsep}{15pt}
        \item $x\dfrac{\text{d}y}{\text{d}x}+y=\cos x$;
        \item $y'+2y=e^x$;
        \item $x+yy'=0$;
        \item $2y''=3y^2$;
        \item $x(y')^2-2yy'+x=0$.
    \end{enumerate}

    \item 检验下列各题中,左边的函数是否为所给微分方程的解:
    \begin{enumerate}[(1)]\setlength{\itemsep}{5pt}\setlength{\topsep}{15pt}
        \item $y=5x^2,\quad xy'=2y$;
        \item $y=3\sin x-4\cos x,\quad y''+y=0$;
        \item $y=e^{-x}+x-1,\quad\dfrac{\text{d}y}{\text{d}x}+y=x$;
        \item $\displaystyle y=e^{x}\int_0^{x}e^{t^2}\text{d}t+Ce^x,\quad y'-y=e^{x+x^2}$;
        \item $x^2+y^2=100,\quad x\text{d}x+y\text{d}y=0$.
    \end{enumerate}


\end{enumerate}

\section{几种常见的一阶微分方程}

\begin{enumerate}\setlength{\itemsep}{7pt}
    \item 求下列微分方程的通解:
    \begin{enumerate}[(1)]\setlength{\itemsep}{5pt}\setlength{\topsep}{15pt}
        \item $\dfrac{\text{d}y}{\text{d}x}=\dfrac{x}{y\sqrt{1-x^2}}$;
        \item $(1+y^2)\text{d}x=x\text{d}y$;
        \item $e^{2x}y\text{d}y-(y+1)\text{d}x=0$;
        \item $e^{-y}(1+y')=1$;
        \item $\sec^2x\tan y\text{d}x+\sec^2y\tan x\text{d}y=0$;
        \item $e^y(1+x^2)\text{d}y-2x(1+e^{y})\text{d}x=0$.
    \end{enumerate}

    \item 求下列微分方程满足所给初值条件的特解:
    \begin{enumerate}[(1)]\setlength{\itemsep}{5pt}\setlength{\topsep}{15pt}
        \item $x^2e^{2y}\text{d}y=(x^3+1)\text{d}x,\;y(1)=0$;
        \item $y'\sin x=y\ln y,\;y\left(\dfrac{\pi}{2}\right)=e$;
        \item $\cos y\text{d}x+(1+e^{-x})\sin y\text{d}y=0,\;y(0)=\dfrac{\pi}{4}$;
        \item $x\text{d}y+2y\text{d}x=0,\;y(2)=1$.
    \end{enumerate}

    \item 求下列微分方程的通解:
    \begin{enumerate}[(1)]\setlength{\itemsep}{5pt}\setlength{\topsep}{15pt}
        \item $x\dfrac{\text{d}y}{\text{d}x}=y(\ln y-\ln x)$;
        \item $(xe^{\frac{y}{x}}+y)\text{d}x-x\text{d}y=0$;
        \item $xy'=\sqrt{x^2-y^2}+y$;
        \item $\left(x\sin\dfrac{y}{x}-y\cos\dfrac{y}{x}\right)\text{d}x+x\cos\dfrac{y}{x}\text{d}y=0$.
    \end{enumerate}

    \item 求下列初值问题的解:
    \begin{enumerate}[(1)]\setlength{\itemsep}{5pt}\setlength{\topsep}{15pt}
        \item $(x^3+y^3)\text{d}x-3xy^2\text{d}y=0,\;y(1)=0$;
        \item $(x+2y)y'=y-2x,\;y(1)=1$;
        \item $\begin{cases}
            (y+\sqrt{x^2+y^2})\text{d}x-x\text{d}y=0\;\;(x>0),\\
            y\big|_{x=1}=0.
        \end{cases}$
    \end{enumerate}

    \item 解下列微分方程:
    \begin{enumerate}[(1)]\setlength{\itemsep}{5pt}\setlength{\topsep}{15pt}
        \item $y'-2xy=e^{x^2}\cos x$;
        \item $y'-y\tan x=\sec x,\;y(0)=0$;
        \item $(y^2+2xy-x)\text{d}y=y^2\text{d}x$;
        \item[*(4)] $y'+\dfrac{y}{x}=2x^{-\frac{1}{2}}y^{\frac{1}{2}}$.
    \end{enumerate}

    \item[*6.] 设有微分方程 $y'-2y=\varphi(x)$,其中 $\varphi(x)=\begin{cases}
        2,&x<1,\\
        0,&x>1.
    \end{cases}$ 试求在 $(-\infty,+\infty)$ 内的连续函数 $y=y(x)$,
    使之在 $x\not=1$ 的区间内都满足方程,且 $y(0)=0$.

\end{enumerate}

\section{高阶微分方程}

\begin{enumerate}\setlength{\itemsep}{7pt}
    \item 求下列各微分方程的通解:
    \begin{enumerate}[(1)]\setlength{\itemsep}{5pt}\setlength{\topsep}{15pt}
        \item $(1+x^2)y''=1$;
        \item $y''+y'=x^2$;
        \item $y''=1+y'^2$;
        \item $y''=(y')^3+y'$.
    \end{enumerate}

    \item 求下列初值问题的解:
    \begin{enumerate}[(1)]\setlength{\itemsep}{5pt}\setlength{\topsep}{15pt}
        \item $y''-2yy'=0,\;y(0)=1,\;y'(0)=1$;
        \item $y^3y''+1=0,\;y(1)=1,\;y'(1)=0$;
        \item $y''-2y'^2=0,\;y(0)=0,\;y'(0)=-1$;
        \item $(1+x^2)y''=2xy',\;y(0)=1,\;y'(0)=3$.
    \end{enumerate}

    \item 设 $y_1(x),\;y_2(x)$ 为二阶线性非齐次微分方程的两个相异的特解,
    求证 $y(x)=y_1(x)-y_2(x)$ 为该方程对应的齐次方程的一个特解.

    \item 求下列齐次线性微分方程的通解:
    \begin{enumerate}[(1)]\setlength{\itemsep}{5pt}\setlength{\topsep}{15pt}
        \item $y''+y'-2y=0$;
        \item $y''-9y=0$;
        \item $y''-4y'+13y=0$;
        \item $y''-2y'+y=0$;
        \item $4y''-8y'+5y=0$;
        \item $4y''+4y'+y=0$;
        \item[**(7)] $y^{(4)}-2y'''+5y''=0$;
        \item[**(8)] $y^{(4)}+2y''+y=0$. 
    \end{enumerate}

    \item 解下列微分方程:
    \begin{enumerate}[(1)]\setlength{\itemsep}{5pt}\setlength{\topsep}{15pt}
        \item $y''+a^2y=e^x$;
        \item $2y''+5y'=5x^2-2x-1$;
        \item $y''-2y'+y=x(1+2e^x)$;
        \item $y''+y=x\cos 2x$;
        \item $y''+3y'+2y=3xe^{-x}$;
        \item $y''+2y'+y=xe^x,\;y\big|_{x=0}=0,\;y'\big|_{x=0}=0$;
        \item $y''+2y'+y=\cos x,\;y\big|_{x=0}=0,\;y'\big|_{x=0}=\dfrac{3}{2}$;
        \item $y''+y=x+\cos x+e^{2x}\cos 3x$.
    \end{enumerate}

    \item[*6.] 设一物体质量为 $m$,以初速 $v_0$ 从一斜面上推下,
    若斜面的倾角为 $\alpha$,摩擦系数为 $\mu$,
    试求物体在斜面上移动的距离和时间的关系.

    \item[*7.] 水平放置的弹簧左端固定,右端与一质量为 $m$ 的物体相连,
    用力将物体从平衡位置 $O$ 向右拉,使弹簧伸长 $a$,然后放开,
    由于弹簧恢复力的作用(劲度系数 $k$),物体便左右振动,
    设摩擦力很小可忽略,求物体的运动规律.

    \item[*8.]
    \begin{enumerate}[(1)]\setlength{\itemsep}{5pt}\setlength{\topsep}{15pt}
        \item 设 $y=e^x(C_1\sin x+C_2\cos x)$ 为某二阶常系数齐次微分方程的通解,求此微分方程;
        \item 已知 $y_1=xe^x+e^{2x},\;y_2=xe^x-e^{-x},\;y_3=xe^x+e^{2x}-e^{-x}$ 
        是某二阶线性常系数非齐次微分方程的三个解,求此微分方程.
    \end{enumerate}

    \item[*9.] 设函数 $f(x),\;g(x)$ 满足 $f'(x)=g(x),\;g'(x)=2e^x-f(x)$,
    且 $f(0)=0,\;g(0)=2$,
    求 $\displaystyle\int_0^{\pi}\left[\dfrac{g(x)}{1+x}-\dfrac{f(x)}{(1+x)^2}\right]\text{d}x$.

    \item[10.] 设函数 $\varphi(x)$ 连续,
    且满足 $\displaystyle\varphi(x)=e^x+\int_0^xt\varphi(t)\text{d}t-x\int_0^x\varphi(t)\text{d}t$,
    求 $\varphi(x)$.

\end{enumerate}

\section{欧拉方程和常系数线性微分方程组}

\begin{enumerate}\setlength{\itemsep}{7pt}
    \item 求下列欧拉方程的通解:
    \begin{enumerate}[(1)]\setlength{\itemsep}{5pt}\setlength{\topsep}{15pt}
        \item $x^2y''+xy'-y=0$;
        \item $y''-\dfrac{y'}{x}+\dfrac{y}{x^2}=\dfrac{2}{x}$;
        \item $x^3y'''+3x^2y''+xy'=24x^2$;
        \item $(2x+1)^2y''-2(2x+1)y'+4y=0$.
    \end{enumerate}

    \item 求下列微分方程组的解:
    \begin{enumerate}[(1)]\setlength{\itemsep}{5pt}\setlength{\topsep}{15pt}
        \item $\begin{cases}
            \dfrac{\text{d}x}{\text{d}t}=x+7y,\\
            \dfrac{\text{d}y}{\text{d}t}=4x-2y;
        \end{cases}$
        \item $\begin{cases}
            \dfrac{\text{d}x}{\text{d}t}+\dfrac{\text{d}y}{\text{d}t}=-x+y+3,\\
            \dfrac{\text{d}x}{\text{d}t}-\dfrac{\text{d}y}{\text{d}t}=x+y-3;
        \end{cases}$
        \item $\begin{cases}
            5\dfrac{\text{d}x}{\text{d}t}-2\dfrac{\text{d}y}{\text{d}t}+2x-y=e^{-t},\\
            \dfrac{\text{d}x}{\text{d}t}+8x-3y=5e^{-t};
        \end{cases}$
        \item $\begin{cases}
            \dfrac{\text{d}x}{\text{d}t}=2x-y+e^t,\\
            \dfrac{\text{d}y}{\text{d}t}=3x-2y-e^{-t},\\
            x(0)=1,\;y(0)=1;
        \end{cases}$
        \item $\begin{cases}
            \dfrac{\text{d}^2x}{\text{d}t^2}=y,\\
            \dfrac{\text{d}^2y}{\text{d}t^2}=x;
        \end{cases}$
        \item $\begin{cases}
            \dfrac{\text{d}^2x}{\text{d}t^2}+2\dfrac{\text{d}y}{\text{d}t}-x=0,\\
            \dfrac{\text{d}x}{\text{d}t}+y=0,\\
            x(0)=1,\;y(0)=0.
        \end{cases}$
    \end{enumerate}


\end{enumerate}

\section{微分方程的应用}

% \begin{enumerate}\setlength{\itemsep}{7pt}
% \end{enumerate}

\section{差分方程简介}

% \begin{enumerate}\setlength{\itemsep}{7pt}
% \end{enumerate}
