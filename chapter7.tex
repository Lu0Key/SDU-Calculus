% !TeX program = XeLaTeX
% !TeX root = main.tex
\chapter{向量代数与空间解析几何}\label{cha:7}

\section{向量及其运算}

\begin{enumerate}\setlength{\itemsep}{7pt}
    \item 设向量 $\boldsymbol{a}, \boldsymbol{b}$ 为非零向量,试作出向量 $2\boldsymbol{a}+\boldsymbol{b}$,$\boldsymbol{a}-2\boldsymbol{b}$,$\boldsymbol{b}-\boldsymbol{a}$,$\dfrac{1}{2}(\boldsymbol{a}+\boldsymbol{b})$ 的图形.
    
    \item 已知向量 $\boldsymbol{a}=(-1, 3, 2),\;\boldsymbol{b}=(2, 5, -1),\;\boldsymbol{c}=(6, 4, -6)$,证明 $\boldsymbol{a}-\boldsymbol{b}$ 与 $\boldsymbol{c}$ 平行.
    
    \item 证明三角形两边中点连线平行于第三边,且等于第三边的一半.
    
    \item 设 $|\boldsymbol{a}|=3,\;|\boldsymbol{b}|=6$,且 $\boldsymbol{a},\;\boldsymbol{b}$ 同方向,求 $\boldsymbol{a}\cdot\boldsymbol{b},\;(\boldsymbol{a}+2\boldsymbol{b})\cdot(2\boldsymbol{a}-\boldsymbol{b})$.
    
    \item 设 $|\boldsymbol{a}|=2,\;|\boldsymbol{b}|=3$,且 $\boldsymbol{a}$ 与 $\boldsymbol{b}$ 垂直,求 $|\boldsymbol{a}\times\boldsymbol{b}|,\;|(\boldsymbol{a}+\boldsymbol{b})\times(2\boldsymbol{a}-\boldsymbol{b})|$.
    
    \item 设 $|\boldsymbol{a}|=2,\;|\boldsymbol{b}|=1,\;(\widehat{\boldsymbol{a}, \boldsymbol{b}})=\dfrac{2\pi}{3}$,求 $2\boldsymbol{a}+\boldsymbol{b}$ 与 $\boldsymbol{a}+4\boldsymbol{b}$ 的夹角.
    
    \item 设 $\boldsymbol{a}+\boldsymbol{b}+\boldsymbol{c}=\boldsymbol{0}$,且 $|\boldsymbol{a}|=1,\;|\boldsymbol{b}|=2,\;|\boldsymbol{c}|=3$,求 $\boldsymbol{a}\cdot\boldsymbol{b}+\boldsymbol{b}\cdot\boldsymbol{c}+\boldsymbol{c}\cdot\boldsymbol{a}$.
    
    \item 一向量的重点 $M_2(4, -2, 0)$,它在三个坐标轴上的投影依次为 $3,\;2,\;7,\;$求该向量的起点 $M_1$.  
    
    \item 设两点 $M_1(2, 0, -3),\;M_2(1, -2, 0)$,在线段 $M_1M_2$ 上求一点 $M$,满足 $M_1M=2MM_2$.
    
    \item 求向量 $\boldsymbol{a}=(1, 1, -4),\;\boldsymbol{b}=(1, -2, 2)$ 的夹角.
    
    \item 设向量 $\boldsymbol{a}=(3, 5, -4),\;b=(2, 1, 8)$,向量 $m\boldsymbol{a}+\boldsymbol{b}$ 与 $z$ 轴垂直,求 $m$.
    
    \item 设向量 $\boldsymbol{a}=3\boldsymbol{i}-\boldsymbol{j}+2\boldsymbol{k},\;\boldsymbol{b}=\boldsymbol{i}+2\boldsymbol{j}-2\boldsymbol{k}$,求
    \begin{enumerate}[(1)]\setlength{\itemsep}{5pt}\setlength{\topsep}{15pt}
        \item $(-2\boldsymbol{a})\cdot\boldsymbol{b}$;
        \item $\boldsymbol{a}\times3\boldsymbol{b}$;
        \item $\cos(\widehat{\boldsymbol{a}, \boldsymbol{b}})$.
    \end{enumerate}

    \item 设向量 $\boldsymbol{a}=-2\boldsymbol{i}+3\boldsymbol{j}+n\boldsymbol{k}$ 与 $\boldsymbol{b}=m\boldsymbol{i}-6\boldsymbol{j}+2\boldsymbol{k}$ 共线,求 $m$ 和 $n$.  
    
    \item 设 $\boldsymbol{a}=3\boldsymbol{i}+4\boldsymbol{k},\;\boldsymbol{b}=-4\boldsymbol{i}+3\boldsymbol{j}$,求
    \begin{enumerate}[(1)]\setlength{\itemsep}{5pt}\setlength{\topsep}{15pt}
        \item 以 $\boldsymbol{a},\;\boldsymbol{b}$ 为邻边的平行四边形的两条对角线的长度;
        \item 以 $\boldsymbol{a},\;\boldsymbol{b}$ 为邻边的平行四边形的面积;
        \item 与 $\boldsymbol{a},\;\boldsymbol{b}$ 垂直的单位向量.
    \end{enumerate}

    \item 设向量 $\boldsymbol{a}=2\boldsymbol{i}-3\boldsymbol{j}+\boldsymbol{k},\;\boldsymbol{b}=\boldsymbol{i}-\boldsymbol{j}+3\boldsymbol{k},\;\boldsymbol{c}=\boldsymbol{i}-2\boldsymbol{j}$,计算
    \begin{enumerate}[(1)]\setlength{\itemsep}{5pt}\setlength{\topsep}{15pt}
        \item $(\boldsymbol{a}\cdot\boldsymbol{b})\boldsymbol{c}-(\boldsymbol{a}\cdot\boldsymbol{c})\boldsymbol{b}$;
        \item $(\boldsymbol{a}\times\boldsymbol{b})\times\boldsymbol{c}$;
        \item $(\boldsymbol{a}+\boldsymbol{b})\times(\boldsymbol{b}+\boldsymbol{c})$;
        \item $(\boldsymbol{a}\times\boldsymbol{b})\cdot\boldsymbol{c}$.
    \end{enumerate}

    \item 判别下列向量 $\boldsymbol{a},\;\boldsymbol{b},\;\boldsymbol{c}$ 是否共面:
    \begin{enumerate}[(1)]\setlength{\itemsep}{5pt}\setlength{\topsep}{15pt}
        \item $\boldsymbol{a}=(3, -2, 1),\;\boldsymbol{b}=(2, 1, 2),\;\boldsymbol{c}=(3, -1, 3)$;
        \item $\boldsymbol{a}=(2, -1, 2),\;\boldsymbol{b}=(1, 2, -3),\;\boldsymbol{c}=(3, -4, 7)$.
    \end{enumerate}

    \item 设 $\boldsymbol{a}=(2, -1, -1),\;\boldsymbol{b}=(1, 1, z)$,问 $z$ 为何值时,$\boldsymbol{a},\;\boldsymbol{b}$ 的夹角 $(\widehat{\boldsymbol{a}, \boldsymbol{b}})$ 最小?并求出此最小值.
\end{enumerate}

\section{空间的平面和直线}

\section{空间的曲面和曲线}





