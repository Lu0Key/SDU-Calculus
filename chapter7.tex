% !TeX program = XeLaTeX
% !TeX root = main.tex
\chapter{向量代数与空间解析几何}\label{cha:7}

\section{向量及其运算}

\begin{enumerate}\setlength{\itemsep}{7pt}
    \item 设向量 $\boldsymbol{a}, \boldsymbol{b}$ 为非零向量,试作出向量 $2\boldsymbol{a}+\boldsymbol{b}$,$\boldsymbol{a}-2\boldsymbol{b}$,$\boldsymbol{b}-\boldsymbol{a}$,$\dfrac{1}{2}(\boldsymbol{a}+\boldsymbol{b})$ 的图形.
    
    \item 已知向量 $\boldsymbol{a}=(-1, 3, 2),\;\boldsymbol{b}=(2, 5, -1),\;\boldsymbol{c}=(6, 4, -6)$,证明 $\boldsymbol{a}-\boldsymbol{b}$ 与 $\boldsymbol{c}$ 平行.
    
    \item 证明三角形两边中点连线平行于第三边,且等于第三边的一半.
    
    \item 设 $|\boldsymbol{a}|=3,\;|\boldsymbol{b}|=6$,且 $\boldsymbol{a},\;\boldsymbol{b}$ 同方向,求 $\boldsymbol{a}\cdot\boldsymbol{b},\;(\boldsymbol{a}+2\boldsymbol{b})\cdot(2\boldsymbol{a}-\boldsymbol{b})$.
    
    \item 设 $|\boldsymbol{a}|=2,\;|\boldsymbol{b}|=3$,且 $\boldsymbol{a}$ 与 $\boldsymbol{b}$ 垂直,求 $|\boldsymbol{a}\times\boldsymbol{b}|,\;|(\boldsymbol{a}+\boldsymbol{b})\times(2\boldsymbol{a}-\boldsymbol{b})|$.
    
    \item 设 $|\boldsymbol{a}|=2,\;|\boldsymbol{b}|=1,\;(\widehat{\boldsymbol{a}, \boldsymbol{b}})=\dfrac{2\pi}{3}$,求 $2\boldsymbol{a}+\boldsymbol{b}$ 与 $\boldsymbol{a}+4\boldsymbol{b}$ 的夹角.
    
    \item 设 $\boldsymbol{a}+\boldsymbol{b}+\boldsymbol{c}=\boldsymbol{0}$,且 $|\boldsymbol{a}|=1,\;|\boldsymbol{b}|=2,\;|\boldsymbol{c}|=3$,求 $\boldsymbol{a}\cdot\boldsymbol{b}+\boldsymbol{b}\cdot\boldsymbol{c}+\boldsymbol{c}\cdot\boldsymbol{a}$.
    
    \item 一向量的重点 $M_2(4, -2, 0)$,它在三个坐标轴上的投影依次为 $3,\;2,\;7,\;$求该向量的起点 $M_1$.  
    
    \item 设两点 $M_1(2, 0, -3),\;M_2(1, -2, 0)$,在线段 $M_1M_2$ 上求一点 $M$,满足 $M_1M=2MM_2$.
    
    \item 求向量 $\boldsymbol{a}=(1, 1, -4),\;\boldsymbol{b}=(1, -2, 2)$ 的夹角.
    
    \item 设向量 $\boldsymbol{a}=(3, 5, -4),\;b=(2, 1, 8)$,向量 $m\boldsymbol{a}+\boldsymbol{b}$ 与 $z$ 轴垂直,求 $m$.
    
    \item 设向量 $\boldsymbol{a}=3\boldsymbol{i}-\boldsymbol{j}+2\boldsymbol{k},\;\boldsymbol{b}=\boldsymbol{i}+2\boldsymbol{j}-2\boldsymbol{k}$,求
    \begin{enumerate}[(1)]\setlength{\itemsep}{5pt}\setlength{\topsep}{15pt}
        \item $(-2\boldsymbol{a})\cdot\boldsymbol{b}$;
        \item $\boldsymbol{a}\times3\boldsymbol{b}$;
        \item $\cos(\widehat{\boldsymbol{a}, \boldsymbol{b}})$.
    \end{enumerate}

    \item 设向量 $\boldsymbol{a}=-2\boldsymbol{i}+3\boldsymbol{j}+n\boldsymbol{k}$ 与 $\boldsymbol{b}=m\boldsymbol{i}-6\boldsymbol{j}+2\boldsymbol{k}$ 共线,求 $m$ 和 $n$.  
    
    \item 设 $\boldsymbol{a}=3\boldsymbol{i}+4\boldsymbol{k},\;\boldsymbol{b}=-4\boldsymbol{i}+3\boldsymbol{j}$,求
    \begin{enumerate}[(1)]\setlength{\itemsep}{5pt}\setlength{\topsep}{15pt}
        \item 以 $\boldsymbol{a},\;\boldsymbol{b}$ 为邻边的平行四边形的两条对角线的长度;
        \item 以 $\boldsymbol{a},\;\boldsymbol{b}$ 为邻边的平行四边形的面积;
        \item 与 $\boldsymbol{a},\;\boldsymbol{b}$ 垂直的单位向量.
    \end{enumerate}

    \item 设向量 $\boldsymbol{a}=2\boldsymbol{i}-3\boldsymbol{j}+\boldsymbol{k},\;\boldsymbol{b}=\boldsymbol{i}-\boldsymbol{j}+3\boldsymbol{k},\;\boldsymbol{c}=\boldsymbol{i}-2\boldsymbol{j}$,计算
    \begin{enumerate}[(1)]\setlength{\itemsep}{5pt}\setlength{\topsep}{15pt}
        \item $(\boldsymbol{a}\cdot\boldsymbol{b})\boldsymbol{c}-(\boldsymbol{a}\cdot\boldsymbol{c})\boldsymbol{b}$;
        \item $(\boldsymbol{a}\times\boldsymbol{b})\times\boldsymbol{c}$;
        \item $(\boldsymbol{a}+\boldsymbol{b})\times(\boldsymbol{b}+\boldsymbol{c})$;
        \item $(\boldsymbol{a}\times\boldsymbol{b})\cdot\boldsymbol{c}$.
    \end{enumerate}

    \item 判别下列向量 $\boldsymbol{a},\;\boldsymbol{b},\;\boldsymbol{c}$ 是否共面:
    \begin{enumerate}[(1)]\setlength{\itemsep}{5pt}\setlength{\topsep}{15pt}
        \item $\boldsymbol{a}=(3, -2, 1),\;\boldsymbol{b}=(2, 1, 2),\;\boldsymbol{c}=(3, -1, 3)$;
        \item $\boldsymbol{a}=(2, -1, 2),\;\boldsymbol{b}=(1, 2, -3),\;\boldsymbol{c}=(3, -4, 7)$.
    \end{enumerate}

    \item 设 $\boldsymbol{a}=(2, -1, -1),\;\boldsymbol{b}=(1, 1, z)$,问 $z$ 为何值时,$\boldsymbol{a},\;\boldsymbol{b}$ 的夹角 $(\widehat{\boldsymbol{a}, \boldsymbol{b}})$ 最小?并求出此最小值.
\end{enumerate}

\section{空间的平面和直线}

\begin{enumerate}\setlength{\itemsep}{7pt}
    \item 求满足下列条件的平面方程:
    \begin{enumerate}[(1)]\setlength{\itemsep}{5pt}\setlength{\topsep}{15pt}
        \item 过点 $M(1, 2, 3)$ 且与平面 $2x+3y+z=0$ 平行;
        \item 过点 $M_1(2, -2, 1),\;M_2(0, 1, 0),\;M_3(1, 4, 5)$ 三点;
        \item 过点 $(4, -3, -2)$ 和点 $(4, 1, 1)$ 且平行于 $x$ 轴.
    \end{enumerate}

    \item 画出下列各平面图形:
    \begin{enumerate}[(1)]\setlength{\itemsep}{5pt}\setlength{\topsep}{15pt}
        \item $2x+3y+4z=6$;
        \item $2x-y=3$;
        \item $x-2y+3z=0$;
        \item $z=2$.
    \end{enumerate}

    \item 求距离原点为 $3$ 且平行于 $x+y+z=1$ 的平面方程.
    
    \item 求三平面 $\pi_1 : x+y+z=4,\;\pi_2 : 3x-y+z=0$ 和 $\pi_3 : x+2y-z=6$ 的交点,以及两两平面之间的夹角.
    
    \item 求满足下列条件的直线方程:
    \begin{enumerate}[(1)]\setlength{\itemsep}{5pt}\setlength{\topsep}{15pt}
        \item 过点 $M_1(-3, 0, 2)$ 和 $M_2(3, 1, 1)$;
        \item 过点 $M(1, 0, 2)$ 且与两直线 $\dfrac{x-1}{1}=y=\dfrac{z+1}{-1}$ 和 $\dfrac{x}{1}=\dfrac{y-1}{-1}=\dfrac{z+1}{0}$ 垂直的直线;
        \item 过点 $M_1(2, -3, 1)$ 与平面 $3x-y+4z-1=0$ 垂直;
        \item 过点 $M_1(0, 2, 4)$ 与两平面 $x+2z-1=0$ 及 $y-3z-2=0$ 都平行;
        \item 过点 $M_1(11, 9, 0)$ 与直线 $\dfrac{x-1}{2}=\dfrac{y+3}{4}=\dfrac{z-5}{5}$ 及直线 $\dfrac{x}{5}=\dfrac{y-2}{-1}=\dfrac{z+1}{2}$ 相交.
    \end{enumerate}

    \item 用对称式方程和参数方程表示直线
    \[
        \begin{cases}
            2x+y-z+1=0,\\
            3x-y-2z-3=0.
        \end{cases}
    \]

    \item 求点 $(2, 0, 1)$ 到直线 $\dfrac{x-5}{3}=\dfrac{y}{2}=\dfrac{z+1}{-1}$ 的距离.
    
    \item 求直线 $\dfrac{x}{-1}=\dfrac{1-y}{-1}=\dfrac{z-1}{2}$ 与平面 $2x+y-zz+4=0$ 的交点和夹角.
    
    \item 判断下列平面与直线间的关系:
    \begin{enumerate}[(1)]\setlength{\itemsep}{5pt}\setlength{\topsep}{15pt}
        \item $\dfrac{x+3}{-2}=\dfrac{y+4}{-7}=\dfrac{z}{3},\;4x-2y-2z-3=0$;
        \item $\dfrac{x}{3}=\dfrac{y}{-2}=\dfrac{z}{7},\;3x-2y+7z-8=0$;
        \item $\dfrac{x-2}{3}=\dfrac{y+2}{1}=\dfrac{z-3}{-4},\;x+y+z-3=0$.
    \end{enumerate}

    \item 问 $k$ 为何值时
    \begin{enumerate}[(1)]\setlength{\itemsep}{5pt}\setlength{\topsep}{15pt}
        \item 直线 $\begin{cases}
            x=kz+2,\\
            y=2kz+4
        \end{cases}$ 与平面 $x+y+z=0$ 平行;
        \item 直线 $\begin{cases}
            x=z+k,\\
            y=z
        \end{cases}$ 与直线 $\begin{cases}
            x=2z+1,\\
            y=3z+2
        \end{cases}$ 相交.
    \end{enumerate}

    \item 求直线 $\begin{cases}
        2x-3y+4z-12=0,\\
        x+4y-2z-10=0
    \end{cases}$ 在平面 $x+y+z-1=0$ 上的投影直线方程.

    \item 在 $z$ 轴上求一点,使它与平面 $12x+9y+20z-19=0$ 和 $16x-12y+15z-9=0$ 等距离.
    
    \item 求点 $M(4, 1, 2)$ 在平面 $x+y+z=1$ 上的投影.
    
    \item 求与平面 $x+6y+z=0$ 平行,且与坐标平面围成的四面体体积为 $6$ 的平面方程.
\end{enumerate}

\section{空间的曲面和曲线}






