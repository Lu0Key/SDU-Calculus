% !TeX program = XeLaTeX
% !TeX root = main.tex
\chapter{曲线积分与曲面积分}\label{cha:10}

% \begin{enumerate}\setlength{\itemsep}{7pt}
% \end{enumerate}

% \begin{enumerate}[(1)]\setlength{\itemsep}{5pt}\setlength{\topsep}{15pt}
% \end{enumerate}

\section{对弧长的曲线积分}

\begin{enumerate}\setlength{\itemsep}{7pt}
    \item 计算下列对弧长的曲线积分:
    \begin{enumerate}[(1)]\setlength{\itemsep}{5pt}\setlength{\topsep}{15pt}
        \item $\displaystyle\int_{L}(x+y)\text{d}s$, 其中 $L$ 为以 $O(0, 0),\;A(1, 0),\;B(0, 1)$ 为顶点的三角形的三条边;
        \item $\displaystyle\int_{L}\sqrt{x^2+y^2}\text{d}s$, 其中 $L$ 是圆周 $x^2+y^2=ax(a>0)$;
        \item $\displaystyle\int_{L}y^2\text{d}s$, 其中 $L$ 为摆线 $x=a(t-\sin t),\;y=a(1-\cos t)(0\leqslant t\leqslant 2\pi)$ 的一拱;
        \item $\displaystyle\int_{L}|y|\text{d}s$, 其中 $L$ 是单位圆的右半圆周,即 $x^2+y^2=1,\;x\geqslant 0$;
        \item $\displaystyle\int_{\Gamma}z\text{d}s$, 其中 $\Gamma$ 为圆锥螺线 $x=t\cos t,\;y=t\sin t,\;z=t(0\leqslant y\leqslant 2)$ 的一段弧;
        \item $\displaystyle\int_{\Gamma}\dfrac{z^2}{x^2+y^2}\text{d}s$, 其中 $\Gamma$ 为圆柱螺线 $x=a\cos t,\;y=a\sin t,\;z=at\;(0\leqslant t\leqslant 2\pi)$ 的一段弧;
        \item $\displaystyle\int_{\Gamma}x^2\text{d}s$, 其中 $\Gamma$ 是球面 $x^2+y^2+z^2=R^2$ 与平面 $x+y+Z=0$ 的交线.
    \end{enumerate}

    \item 求下列空间曲线段的弧长:
    \begin{enumerate}[(1)]\setlength{\itemsep}{5pt}\setlength{\topsep}{15pt}
        \item $x=3t,\;y=3t^2,\;z=2t^3$, 从点 $O(0, 0, 0)$ 到点 $A(3, 3, 2)$;
        \item $x=e^{-t}\cos y,\;y=e^{-t}\sin t,\;z=e^{-t},\;0\leqslant t<+\infty$.
    \end{enumerate}
\end{enumerate}


\section{对坐标的曲线积分}
\begin{enumerate}\setlength{\itemsep}{7pt}
    \item 设 $L$ 为 $x$ 轴上从点 $A(a, 0)$ 到点 $B(b, 0)$ 的一段直线,证明
    \[
        \int_{L}P(x, y)\text{d}x = \int_{a}^{b}P(x, 0)\text{d}x.
    \]
    依此,您能说明第二类曲线积分与定积分的关系吗?

    \item 计算下列第二类曲线积分:
    \begin{enumerate}[(1)]\setlength{\itemsep}{5pt}\setlength{\topsep}{15pt}
        \item $\displaystyle\int_{L}(x^2-y^2)\text{d}x$,$L$ 是一个抛物线 $y=x^2$ 上从点 $(0, 0)$ 到 $(2, 4)$ 的一段弧;
        \item $\displaystyle\int_{L}(x^2-2xy)\text{d}x+(y^2-2xy)\text{d}y$,$L$ 是抛物线 $y=x^2$ 上从点 $A(-1, 1)$ 到 $B(1, 1)$ 的一段弧;
        \item $\displaystyle\oint_{L}xy\text{d}x$,$L$ 为圆周 $x^2+y^2=2ax(a>0)$ 与 $x$ 轴围成的第一象限的区域的整个边界(按逆时针方向绕行);
        \item $\displaystyle\oint_{L}(x^2+y^2)\text{d}y$,$L$ 是由直线 $x=1,\;y=1,\;x=3,\;y=5$ 构成的正向矩形闭回路;
        \item $\displaystyle\int_{L}(2a-y)\text{d}x-(a-y)\text{d}y$,$L$ 为摆线 $x=a(t-\sin t),\;y=a(1-\cos t)$ 的一供(对应于由 $t_1=0$ 到 $t_2=2\pi$ 的一段弧);
        \item $\displaystyle\int_{\Gamma}y\text{d}x+z\text{d}y+x\text{d}z$,$\Gamma$ 为曲线 $x=a\cos t,\;y=a\sin t,\;z=bt$ 上从 $t=0$ 到 $t=2\pi$ 的一段弧;
        \item $\displaystyle\int_{\Gamma}x\text{d}x+y\text{d}y+(x+y-1)\text{d}z$,$\Gamma$ 是从点 $(1, 1, 1)$ 到点 $(2, 3, 4)$ 的一段直线;
        \item $\displaystyle\oint_{L}\dfrac{(x+y)\text{d}x-(x-y)\text{d}y}{x^2+y^2}$,$L$ 为依逆时针方向沿圆 $x^2+y^2=a^2$ 绕行一周的路径.
    \end{enumerate}

    \item 将对坐标的曲线积分 $\displaystyle\int_{L}P(x, y)\text{d}x+Q(x, y)\text{d}y$ 化为对弧长的曲线积分,其中 $L$ 为沿抛物线 $y=x^2$ 从点 $(0, 0)$ 到点 $(1, 1)$ 的一段弧.
    
    \item 设 $\Gamma$ 为曲线 $x=t,\;y=t^2,\;z=t^3$ 上相应于 $t$ 从 $0$ 变到 $1$ 的曲线弧,把对坐标的曲线积分 $\displaystyle\int_{L}P\text{d}x+Q\text{d}y+R\text{d}z$ 化成对弧长的曲线积分.
    
    \item 一力场由沿 $x$ 轴正方向的常力 $\boldsymbol{F}$ 所构成,试求当一质量为 $m$ 的质点沿圆周 $x^2+y^2=R^2$ 按逆时针方向移过位于第一象限的那一段弧时场力所做的功.
    
    \item 设有一质量为 $m$ 的质点受重力作用在铅直平面上沿某一光滑曲线 $L$ 从点 $A$ 移动到点 $B$,求重力所做的功(图10.2.5).(估计这题不会去做,Tikz还不怎么会用).
    
    \item[*7.] 在变力 $\boldsymbol{F}=yz\boldsymbol{i}+zx\boldsymbol{j}+zy\boldsymbol{k}$ 作用下,质点由原点沿直线运动到椭球面 $\dfrac{x^2}{a^2}+\dfrac{y^2}{b^2}+\dfrac{z^2}{c^2}=1$ 上第一卦限的点 $M(\xi, \eta, \zeta)$. 
    问当 $\xi,\;\eta,\;\zeta$ 取何值时,力 $\boldsymbol{F}$ 所作的功 $W$ 最大?并求出 $W$ 的最大值.
\end{enumerate}

\section{格林公式及其应用}

\section{对面积的曲面积分}

\section{对坐标的曲面积分}

\section{高斯公式和斯托克斯公式}

\section{场论简介}

\section{用MATLAB计算曲线积分和曲面积分}