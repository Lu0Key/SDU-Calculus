% !TeX program = XeLaTeX
% !TeX root = main.tex
\chapter{曲线积分与曲面积分}\label{cha:10}

% \begin{enumerate}\setlength{\itemsep}{7pt}
% \end{enumerate}

% \begin{enumerate}[(1)]\setlength{\itemsep}{5pt}\setlength{\topsep}{15pt}
% \end{enumerate}

\section{对弧长的曲线积分}

\begin{enumerate}\setlength{\itemsep}{7pt}
    \item 计算下列对弧长的曲线积分:
    \begin{enumerate}[(1)]\setlength{\itemsep}{5pt}\setlength{\topsep}{15pt}
        \item $\displaystyle\int_{L}(x+y)\text{d}s$, 其中 $L$ 为以 $O(0, 0),\;A(1, 0),\;B(0, 1)$ 为顶点的三角形的三条边;
        \item $\displaystyle\int_{L}\sqrt{x^2+y^2}\text{d}s$, 其中 $L$ 是圆周 $x^2+y^2=ax(a>0)$;
        \item $\displaystyle\int_{L}y^2\text{d}s$, 其中 $L$ 为摆线 $x=a(t-\sin t),\;y=a(1-\cos t)(0\leqslant t\leqslant 2\pi)$ 的一拱;
        \item $\displaystyle\int_{L}|y|\text{d}s$, 其中 $L$ 是单位圆的右半圆周,即 $x^2+y^2=1,\;x\geqslant 0$;
        \item $\displaystyle\int_{\Gamma}z\text{d}s$, 其中 $\Gamma$ 为圆锥螺线 $x=t\cos t,\;y=t\sin t,\;z=t(0\leqslant y\leqslant 2)$ 的一段弧;
        \item $\displaystyle\int_{\Gamma}\dfrac{z^2}{x^2+y^2}\text{d}s$, 其中 $\Gamma$ 为圆柱螺线 $x=a\cos t,\;y=a\sin t,\;z=at\;(0\leqslant t\leqslant 2\pi)$ 的一段弧;
        \item $\displaystyle\int_{\Gamma}x^2\text{d}s$, 其中 $\Gamma$ 是球面 $x^2+y^2+z^2=R^2$ 与平面 $x+y+Z=0$ 的交线.
    \end{enumerate}

    \item 求下列空间曲线段的弧长:
    \begin{enumerate}[(1)]\setlength{\itemsep}{5pt}\setlength{\topsep}{15pt}
        \item $x=3t,\;y=3t^2,\;z=2t^3$, 从点 $O(0, 0, 0)$ 到点 $A(3, 3, 2)$;
        \item $x=e^{-t}\cos y,\;y=e^{-t}\sin t,\;z=e^{-t},\;0\leqslant t<+\infty$.
    \end{enumerate}
\end{enumerate}


\section{对坐标的曲线积分}

\section{格林公式及其应用}

\section{对面积的曲面积分}

\section{对坐标的曲面积分}

\section{高斯公式和斯托克斯公式}

\section{场论简介}

\section{用MATLAB计算曲线积分和曲面积分}