% !TeX program = XeLaTeX
% !TeX root = main.tex
\chapter{曲线积分与曲面积分}\label{cha:10}

% \begin{enumerate}\setlength{\itemsep}{7pt}
% \end{enumerate}

% \begin{enumerate}[(1)]\setlength{\itemsep}{5pt}\setlength{\topsep}{15pt}
% \end{enumerate}

\section{对弧长的曲线积分}

\begin{enumerate}\setlength{\itemsep}{7pt}
    \item 计算下列对弧长的曲线积分:
    \begin{enumerate}[(1)]\setlength{\itemsep}{5pt}\setlength{\topsep}{15pt}
        \item $\displaystyle\int_{L}(x+y)\text{d}s$, 其中 $L$ 为以 $O(0, 0),\;A(1, 0),\;B(0, 1)$ 为顶点的三角形的三条边;
        \item $\displaystyle\int_{L}\sqrt{x^2+y^2}\text{d}s$, 其中 $L$ 是圆周 $x^2+y^2=ax(a>0)$;
        \item $\displaystyle\int_{L}y^2\text{d}s$, 其中 $L$ 为摆线 $x=a(t-\sin t),\;y=a(1-\cos t)(0\leqslant t\leqslant 2\pi)$ 的一拱;
        \item $\displaystyle\int_{L}|y|\text{d}s$, 其中 $L$ 是单位圆的右半圆周,即 $x^2+y^2=1,\;x\geqslant 0$;
        \item $\displaystyle\int_{\Gamma}z\text{d}s$, 其中 $\Gamma$ 为圆锥螺线 $x=t\cos t,\;y=t\sin t,\;z=t(0\leqslant y\leqslant 2)$ 的一段弧;
        \item $\displaystyle\int_{\Gamma}\dfrac{z^2}{x^2+y^2}\text{d}s$, 其中 $\Gamma$ 为圆柱螺线 $x=a\cos t,\;y=a\sin t,\;z=at\;(0\leqslant t\leqslant 2\pi)$ 的一段弧;
        \item $\displaystyle\int_{\Gamma}x^2\text{d}s$, 其中 $\Gamma$ 是球面 $x^2+y^2+z^2=R^2$ 与平面 $x+y+Z=0$ 的交线.
    \end{enumerate}

    \item 求下列空间曲线段的弧长:
    \begin{enumerate}[(1)]\setlength{\itemsep}{5pt}\setlength{\topsep}{15pt}
        \item $x=3t,\;y=3t^2,\;z=2t^3$, 从点 $O(0, 0, 0)$ 到点 $A(3, 3, 2)$;
        \item $x=e^{-t}\cos y,\;y=e^{-t}\sin t,\;z=e^{-t},\;0\leqslant t<+\infty$.
    \end{enumerate}
\end{enumerate}


\section{对坐标的曲线积分}
\begin{enumerate}\setlength{\itemsep}{7pt}
    \item 设 $L$ 为 $x$ 轴上从点 $A(a, 0)$ 到点 $B(b, 0)$ 的一段直线,证明
    \[
        \int_{L}P(x, y)\text{d}x = \int_{a}^{b}P(x, 0)\text{d}x.
    \]
    依此,您能说明第二类曲线积分与定积分的关系吗?

    \item 计算下列第二类曲线积分:
    \begin{enumerate}[(1)]\setlength{\itemsep}{5pt}\setlength{\topsep}{15pt}
        \item $\displaystyle\int_{L}(x^2-y^2)\text{d}x$,$L$ 是一个抛物线 $y=x^2$ 上从点 $(0, 0)$ 到 $(2, 4)$ 的一段弧;
        \item $\displaystyle\int_{L}(x^2-2xy)\text{d}x+(y^2-2xy)\text{d}y$,$L$ 是抛物线 $y=x^2$ 上从点 $A(-1, 1)$ 到 $B(1, 1)$ 的一段弧;
        \item $\displaystyle\oint_{L}xy\text{d}x$,$L$ 为圆周 $x^2+y^2=2ax(a>0)$ 与 $x$ 轴围成的第一象限的区域的整个边界(按逆时针方向绕行);
        \item $\displaystyle\oint_{L}(x^2+y^2)\text{d}y$,$L$ 是由直线 $x=1,\;y=1,\;x=3,\;y=5$ 构成的正向矩形闭回路;
        \item $\displaystyle\int_{L}(2a-y)\text{d}x-(a-y)\text{d}y$,$L$ 为摆线 $x=a(t-\sin t),\;y=a(1-\cos t)$ 的一供(对应于由 $t_1=0$ 到 $t_2=2\pi$ 的一段弧);
        \item $\displaystyle\int_{\Gamma}y\text{d}x+z\text{d}y+x\text{d}z$,$\Gamma$ 为曲线 $x=a\cos t,\;y=a\sin t,\;z=bt$ 上从 $t=0$ 到 $t=2\pi$ 的一段弧;
        \item $\displaystyle\int_{\Gamma}x\text{d}x+y\text{d}y+(x+y-1)\text{d}z$,$\Gamma$ 是从点 $(1, 1, 1)$ 到点 $(2, 3, 4)$ 的一段直线;
        \item $\displaystyle\oint_{L}\dfrac{(x+y)\text{d}x-(x-y)\text{d}y}{x^2+y^2}$,$L$ 为依逆时针方向沿圆 $x^2+y^2=a^2$ 绕行一周的路径.
    \end{enumerate}

    \item 将对坐标的曲线积分 $\displaystyle\int_{L}P(x, y)\text{d}x+Q(x, y)\text{d}y$ 化为对弧长的曲线积分,其中 $L$ 为沿抛物线 $y=x^2$ 从点 $(0, 0)$ 到点 $(1, 1)$ 的一段弧.
    
    \item 设 $\Gamma$ 为曲线 $x=t,\;y=t^2,\;z=t^3$ 上相应于 $t$ 从 $0$ 变到 $1$ 的曲线弧,把对坐标的曲线积分 $\displaystyle\int_{L}P\text{d}x+Q\text{d}y+R\text{d}z$ 化成对弧长的曲线积分.
    
    \item 一力场由沿 $x$ 轴正方向的常力 $\boldsymbol{F}$ 所构成,试求当一质量为 $m$ 的质点沿圆周 $x^2+y^2=R^2$ 按逆时针方向移过位于第一象限的那一段弧时场力所做的功.
    
    \item 设有一质量为 $m$ 的质点受重力作用在铅直平面上沿某一光滑曲线 $L$ 从点 $A$ 移动到点 $B$,求重力所做的功(图10.2.5).(估计这题不会去做,Tikz还不怎么会用).
    
    \item[*7.] 在变力 $\boldsymbol{F}=yz\boldsymbol{i}+zx\boldsymbol{j}+zy\boldsymbol{k}$ 作用下,质点由原点沿直线运动到椭球面 $\dfrac{x^2}{a^2}+\dfrac{y^2}{b^2}+\dfrac{z^2}{c^2}=1$ 上第一卦限的点 $M(\xi, \eta, \zeta)$. 
    问当 $\xi,\;\eta,\;\zeta$ 取何值时,力 $\boldsymbol{F}$ 所作的功 $W$ 最大?并求出 $W$ 的最大值.
\end{enumerate}

\section{格林公式及其应用}

\begin{enumerate}\setlength{\itemsep}{7pt}
    \item 计算下列各曲线积分:
    \begin{enumerate}[(1)]\setlength{\itemsep}{5pt}\setlength{\topsep}{15pt}
        \item $\displaystyle\oint_{L}(2xy-x^2)\text{d}x+(x+y^2)\text{d}y$,$L$ 是由抛物线 $y=x^2$ 和 $y^2=x$ 所围成的区域的正项边界曲线;
        \item $\displaystyle\oint_{L}(x-y+2)\text{d}x+(x^2-xy+y^2)\text{d}y$,$L$ 是按顺时针方向绕椭圆 $\dfrac{x^2}{a^2}+\dfrac{y^2}{b^2}=1$ 一周的路径;
        \item $\displaystyle\int_{L}(ye^x+2x+1)\text{d}x-(2y-e^x)\text{d}y$,$L$ 是由远点 $(0, 0)$ 沿上半圆周 $y=\sqrt{2x-x^2}$ 到点 $A(1, 1)$ 的一段弧;
        \item $\displaystyle\int_{L}(y-x)\text{d}x+(x^2+y^2-\arctan y)\text{d}y$,$L$ 是由点 $A(1, -1)$ 沿抛物线 $y^2=x$ 到点 $B(1, 1)$ 的一段弧;
        \item $\displaystyle\int_{L}(x^2-e^x\cos y)\text{d}x+(e^x\sin y+3x)\text{d}y$,$L$ 是由点 $A(0, 2)$ 沿半圆周 $x=\sqrt{2y-y^2}$ 到原点的一段弧.
    \end{enumerate}

    \item 计算 $\displaystyle\oint_{L}\dfrac{-y\text{d}x+x\text{d}y}{x^2+y^2}$,其中路径 $L$ 为一下三种情况:
    \begin{enumerate}[(1)]\setlength{\itemsep}{5pt}\setlength{\topsep}{15pt}
        \item $L_1$ 为不过原点的正向闭曲线,且 $L_1$ 围成的区域 $D$ 不含原点;
        \item $L_2$ 为圆周 $x^2+y^2=a^2$ 的逆时针方向;
        \item $L_3$ 为包含 $L_2$ 的任一条闭曲线,取逆时针方向.
    \end{enumerate}
    从积分沿以上三条路径的积分值,您可以得到关于这个积分的更一般的结果吗?

    \item 设有一个变力 $\boldsymbol{F}=(x+y^2)\boldsymbol{i}+(2xy-8)\boldsymbol{j}$,这变力确定了一个力场,证明质点在此场内移动时,场力所作的功与路径无关.
    
    \item 已知点 $O(0, 0)$ 及点 $A(1, 1)$,
    且积分 $\displaystyle I=\int_{\overset{\large{\frown}}{OA}}(ax\cos y-y^2\sin x)\text{d}x+(by\cos x-x^2\sin y)\text{d}y$ 与路径无关,求 $a, b$ 及 $I$ 的值.
    
    \item 设 $f(u)$ 在 $(-\infty, +\infty)$ 内有连续导数,$L$ 为 $xOy$ 平面上任意一条光滑的闭曲线,证明:
    \[
        \oint_{L}f(xy)(y\text{d}x+x\text{d}y) = 0.
    \]

    \item 验证下列 $P(x, y)\text{d}x+Q(x, y)\text{d}y$ 在整个 $xOy$ 平面上是某一个函数 $u(x, y)$ 全微分,并求它的一个原函数:
    \begin{enumerate}[(1)]\setlength{\itemsep}{5pt}\setlength{\topsep}{15pt}
        \item $(3x^2y+xy^2)\text{d}x+(x^3+x^2y-e^y+1)\text{d}y$;
        \item $4\sin x\cdot\sin 3y\cdot\cos x\text{d}x-3\cos3y\cdot\cos2x\text{d}y$.
    \end{enumerate}

    \item 求表达式 $3x^2y^2\text{d}x+2x^2y\text{d}y$ 的一个原函数,并计算积分
    \[
        I=\int_{(-1, 1)}^{(2, -2)}3x^2y^2\text{d}x+2x^2y\text{d}y.
    \]

    \item[*8.] 设曲线积分 $\displaystyle\int_{L}[\sin x-f(x)]\dfrac{y}{x}\text{d}x+f(y)\text{d}y$ 与积分路径无关,且 $f(\pi)$,求 $f(x)$,并计算 $L$ 始点为 $A(1, 0)$,终点 $B(\pi, \pi)$ 时曲线积分的值.
    
    \item[*9.] 设函数 $f(x)$ 在 $(-\infty, +\infty)$ 内具有一阶连续导数,$L$ 是上半平面 $(y>0)$ 内的有向分段光滑曲线,其起点 $(a, b)$,终点为 $(c, d)$,记
    \[
        I=\int_{L}\dfrac{1}{y}[1+y^2f(xy)]\text{d}x+\dfrac{x}{y^2}[y^2f(xy)-1]\text{d}y,
    \]
    \begin{enumerate}[(1)]\setlength{\itemsep}{5pt}\setlength{\topsep}{15pt}
        \item 证明曲线积分 $I$ 与路径 $L$ 无关;
        \item 当 $ab=cd$ 时,求 $I$ 的值.
    \end{enumerate}

    \item[*10.] 求 $\displaystyle I=\int_{L}[e^x\sin y-b(x+y)]\text{d}x+[2^x\cos y-ax]\text{d}y$,其中 $a, b$ 为正的常数,$L$ 为从点 $A(2a, 0)$ 沿曲线 $y=\sqrt{2a-x^2}$ 到点 $Q(0, 0)$ 的弧.
    
    \item[11.] 求下列各微分方程的通解:
    \begin{enumerate}[(1)]\setlength{\itemsep}{5pt}\setlength{\topsep}{15pt}
        \item $\displaystyle(x^2+2xy-y^2)\text{d}x+(x^2-2xy-y^2)\text{d}y=0$;
        \item $\displaystyle[\cos(x+y^2)+2y]\text{d}x+[2y\cos(x+y^2)+3x]\text{d}y=0$;
        \item[* (3)] $\displaystyle(x^2+y^2+2x)\text{d}x+2y\text{d}y=0$;
        \item[* (4)] $\displaystyle y\text{d}x-x\text{d}y+y^2x\text{d}x=0.$ 
    \end{enumerate} 

\end{enumerate}


\section{对面积的曲面积分}

\begin{enumerate}\setlength{\itemsep}{7pt}
    \item 计算下列各曲面积分:
    \begin{enumerate}[(1)]\setlength{\itemsep}{5pt}\setlength{\topsep}{15pt}
        \item $\displaystyle\stackrel{\iint}{\Sigma}(x+y+z)\text{d}S$,其中 $\Sigma$ 为上半球面 $z=\sqrt{a^2-x^2-y^2}$;
        \item $\displaystyle\iint\limits_{\Sigma}\dfrac{1}{(1+x+y)^2}\text{d}S$,其中 $\Sigma$ 是平面 $x+y+z=1$ 在第一卦限的部分;
        \item $\displaystyle\iint\limits_{\Sigma}(xy+yz+zx)\text{d}S$,其中 $\Sigma$ 为圆锥面 $z=\sqrt{x^2+y^2}$ 被圆柱面 $x^2+y^2=2ax$ 所截图部分;
        \item $\displaystyle\iint\limits_{\Sigma}(x^2+z^2)\text{d}S$,其中 $\Sigma$ 为圆锥面 $y=\sqrt{x^2+z^2}$ 被 $y=2$ 截下的有限部分.
    \end{enumerate}

    \item 设曲面 $\Sigma$ 是 $xOy$ 平面的一个有界闭区域,问:曲面积分 $\displaystyle\iint\limits_{\Sigma}f(x, y, z)\text{d}S$ 与二重积分有什么关系?
    
    \item 已知半径为 $a$ 的球面上每一点的面密度 $\rho$ 等于该点到某一确定直径的距离,试求此球面的质量.
    
    \item 设锥面 $z=\sqrt{x^2+y^2}$,圆锥面 $x^2+y^2=2ax(a>0)$,求:
    \begin{enumerate}[(1)]\setlength{\itemsep}{5pt}\setlength{\topsep}{15pt}
        \item 锥面被柱面所截部分的面积;
        \item 圆锥面被锥面和 $xOy$ 坐标平面所截部分的面积.
    \end{enumerate}

    \item[*5.] 设 $S$ 为椭球面 $\dfrac{x^2}{2}+\dfrac{y^2}{2}+z^2=1$ 的上半部分,点 $P(x, y, z)\in S$,平面 $\pi$ 为 $S$ 在点 $P$ 处的切平面,$\rho(x, y, z)$ 为点 $O(0, 0, 0)$ 到平面 $\pi$ 的距离,求 $\displaystyle\iint\limits_{S}\dfrac{z}{\rho(x, y, z)}\text{d}S$.
    
    \item[*6.] 设半径为 $R$ 的球面的球心在半径为 $a$ 的定球面上,试问当 $R$ 为何值时,前者夹在定球面内的表面积为最大.
    
    \item[*7.] 设在海湾中,海潮的高潮与低潮之间的差是 $2m$. 一个岛的陆地高度 $z=30\left(1-\dfrac{x^2+y^2}{10^6}\right)$(单位为 $m$),并设水平面 $z=0$ 对应于低潮的位置. 求高潮与低潮时小岛露出水面的面积之比. 
\end{enumerate}

\section{对坐标的曲面积分}

\begin{enumerate}\setlength{\itemsep}{7pt}
    \item 当 $\Sigma$ 为 $xOy$ 平面内的一个闭区域时,第二类曲面积分 $\displaystyle\iint\limits_{\Sigma}R(x, y, z)\text{d}x\text{d}y$ 与二重积分有什么关系?第一类曲面积分 $\displaystyle\iint\limits_{\Sigma}R(x, y, z)\text{d}S$ 与二重积分有什么关系?
    
    \item 计算下列曲面积分:
    \begin{enumerate}[(1)]\setlength{\itemsep}{5pt}\setlength{\topsep}{15pt}
        \item $\displaystyle\iint\limits_{\Sigma}(x+2y+3z)\text{d}y\text{d}z$,$\Sigma$ 为柱面 $y^2+z^2=1(1\leqslant x\leqslant2)$ 的外侧;
        \item $\displaystyle\iint\limits_{\Sigma}xy^2(z-1)^3\text{d}x\text{d}y$,$\Sigma$ 是平面 $z=1$ 被柱面 $x^2+y^2=a^2$ 所截部分的上侧;
        \item $\displaystyle\oiint\limits_{\Sigma}(x^2+y^2+z^2)\text{d}x\text{d}y$,$\Sigma$ 是球面 $x^2+y^2+z^2=a^2$ 的外侧.
    \end{enumerate}

    \item 计算下列曲面积分
    \begin{enumerate}[(1)]\setlength{\itemsep}{5pt}\setlength{\topsep}{15pt}
        \item $\displaystyle\iint\limits_{\Sigma}z\text{d}x\text{d}y+x\text{d}y\text{d}z+y\text{d}z\text{d}x$,$\Sigma$ 是柱面 $x^2+y^2=1$ 被平面 $z=0$ 及 $z=3$ 所截得的在第一卦限内的部分的前侧;
        \item $\displaystyle\oiint\limits_{\Sigma}y^2z\text{d}x\text{d}y$,封闭曲面 $\Sigma$ 为旋转抛物面 $z=x^2+Y^2$ 与平面 $z=1$ 所围空间体表面的外侧;
        \item $\displaystyle\oiint\limits_{\Sigma}x\text{d}y\text{d}z$,其中 $\Sigma$ 与(2)题相同.
    \end{enumerate}

    \item 已知流速场 $v(x, y, z)=x^2\boldsymbol{i}+y^2\boldsymbol{j}+z^2\boldsymbol{k}$,封闭曲面 $\Sigma$ 为平面 $x+y+z=1$ 与三个坐标平面所围成的四面体的表面,试求流速场由曲面 $\Sigma$ 的内部流向其外部的流量 $\Phi$.
    
    \item 将对坐标的曲面积分
    \[
        \iint\limits_{\Sigma}P(x, y, z)\text{d}y\text{d}z+Q(x, y, z)\text{d}z\text{d}x+R(x, y, z)\text{d}x\text{d}y
    \]
    化为对面积的曲面积分,其中 $\Sigma$ 是:
    \begin{enumerate}[(1)]\setlength{\itemsep}{5pt}\setlength{\topsep}{15pt}
        \item 平面 $3x+2y+2\sqrt{3}z=6$ 在第一卦限部分的上侧;
        \item $y^2=2-x(y\geqslant0)$ 的前侧;
        \item $z=xy$ 位于第一卦限部分的上侧.
    \end{enumerate}    
    
\end{enumerate}

\section{高斯公式和斯托克斯公式}

% \begin{enumerate}\setlength{\itemsep}{7pt}
% \end{enumerate}

\section{场论简介}

% \begin{enumerate}\setlength{\itemsep}{7pt}
% \end{enumerate}

\section{用MATLAB计算曲线积分和曲面积分}