% !TeX program = XeLaTeX
% !TeX root = main.tex
\chapter{多元函数微分学及其应用}\label{cha:8}

% \begin{enumerate}\setlength{\itemsep}{7pt}
% \end{enumerate}

% \begin{enumerate}[(1)]\setlength{\itemsep}{5pt}\setlength{\topsep}{15pt}
% \end{enumerate}

\section{多元函数的概念及其极限和连续}

\begin{enumerate}\setlength{\itemsep}{7pt}
    \item 求下列函数的定义域:
    \begin{enumerate}[(1)]\setlength{\itemsep}{5pt}\setlength{\topsep}{15pt}
        \item $z=\sqrt{\sin(x^2+y^2)}$;
        \item $z=\ln(x^2-3y+2)$;
        \item $z=\sqrt{R^2-x^2-y^2}+\dfrac{1}{\sqrt{x^2+y^2-r^2}}\;\;(R>r>0)$;
        \item $z=\sqrt{x-\sqrt{y}}$;
        \item $z=\dfrac{1}{\sqrt{x+y}}+\dfrac{1}{\sqrt{x-y}}$;
        \item $z=\dfrac{xy}{x^2+y^2}$;
        \item $z=\ln(x-y)-\dfrac{\sqrt{y}}{\sqrt{1-x^2-y^2}}$;
        \item $z=\sqrt{1-x^2}+\sqrt{y^2-4}$.
    \end{enumerate}

    \item 若 $f(x, y)=\dfrac{2xy}{x^2+y^2}$,求 $f\left(1,\dfrac{y}{x}\right)$.
    
    \item 设 $f\left(x+y, \dfrac{y}{x}\right)=x^2-y^2$,求 $f(x, y)$.
    
    \item 设 $z=x+y+f(x-y)$,且当 $y=0$ 时,$z=x^2$,求函数 $f(x)$ 和 $z$ 的表达式.
    
    \item 求下列函数的间断点:
    \begin{enumerate}[(1)]\setlength{\itemsep}{5pt}\setlength{\topsep}{15pt}
        \item $z=\dfrac{x+1}{\sqrt{x^2+y^2}}$;
        \item $z=\dfrac{xy^2}{x+y}$;
        \item $z=\ln(a^2-x^2-y^2)$;
        \item $z=\dfrac{1}{\sin x\cdot\sin y}$.
    \end{enumerate}

    \item 求下列函数的极限:
    \begin{enumerate}[(1)]\setlength{\itemsep}{5pt}\setlength{\topsep}{15pt}
        \item $\displaystyle \lim_{(x, y)\to(0, 1)}\dfrac{\tan(xy)}{x}$;
        \item $\displaystyle \lim_{(x, y)\to(0, 0)}\dfrac{xy}{\sqrt{xy+1}-1}$;
        \item $\displaystyle \lim_{(x, y)\to(0, 2)}\dfrac{\sin(xy)}{x}$;
        \item $\displaystyle \lim_{(x, y)\to(0, 1)}\dfrac{1-xy}{x^2+y^2}$.
    \end{enumerate}

    \item[**7.] 讨论二元函数
    \[
        f(x, y)=\begin{cases}
            \dfrac{xy}{x^2+y^2},&x^2+y^2\not=0,\\
            0,&x^2+y^2=0
        \end{cases}
    \]
    在点 $(0, 0)$ 的连续性.  
\end{enumerate}

\section{偏导数与全微分}

\begin{enumerate}\setlength{\itemsep}{7pt}
    \item 求下列函数在给定点处的偏导数:
    \begin{enumerate}[(1)]\setlength{\itemsep}{5pt}\setlength{\topsep}{15pt}
        \item $z=x^2+3xy+y^2$,求 $z_x'(1, 2),\;z_y'(1, 2)$;
        \item $z=e^{x^2+y^2}$,求 $z_x'(0, 1),\;z_y'(0, 1)$;
        \item $z=\dfrac{xy(x^2-y^2)}{x^2+y^2}$,求 $z_x'(1, 1),\;z_y'(1, 1)$;
        \item $z=\ln|xy|$,求 $z_x'(-1, -1),\;z_y'(1, 1)$.
    \end{enumerate}

    \item 求下列函数的一阶偏导数:
    \begin{enumerate}[(1)]\setlength{\itemsep}{5pt}\setlength{\topsep}{15pt}
        \item $z=\sin(xy)+\cos^2(xy)$;
        \item $z=\ln(z+\ln y)$;
        \item $z=x^2\arctan\dfrac{y}{x}-y^2\arctan\dfrac{x}{y}$;
        \item $z=x\cdot\ln\dfrac{y}{x}$;
        \item $z=\arcsin\dfrac{x}{y}$;
        \item $z=e^{\frac{x}{y}}+e^{\frac{y}{x}}$;
        \item $z=(1+xy)^{y}$;
        \item $z=\ln\dfrac{\sqrt{x^2+y^2}-x}{\sqrt{x^2+y^2}+x}$
    \end{enumerate}

    \item 求下列函数的二阶偏导数:
    \begin{enumerate}[(1)]\setlength{\itemsep}{5pt}\setlength{\topsep}{15pt}
        \item $z=x\ln(xy)$;
        \item $z=x^4+y^4-4x^2y^2$;
        \item $z=\arctan\dfrac{y}{x}$;
        \item $z=y^x$.
    \end{enumerate}

    \item 求下列函数的全微分:
    \begin{enumerate}[(1)]\setlength{\itemsep}{5pt}\setlength{\topsep}{15pt}
        \item $z=e^{x(x^2+y^2)}$;
        \item $z=\arctan\dfrac{x+y}{x-y}$;
        \item $z=\sqrt{\dfrac{y}{x}}$;
        \item $z=\sqrt{\ln(xy)}$.
    \end{enumerate}

    \item 证明下列各题:
    \begin{enumerate}[(1)]\setlength{\itemsep}{5pt}\setlength{\topsep}{15pt}
        \item 设 $T=2\pi\sqrt{\dfrac{l}{g}}$, 证明 $l\dfrac{\partial T}{\partial l}+g\dfrac{\partial T}{\partial g}=0$;
        \item 设 $z=x^y\cdot y^x$, 证明 $x\dfrac{\partial z}{\partial x}+y\dfrac{\partial z}{\partial y}=z(x+y+\ln z)$;
        \item 设 $z=f(ax+by)$, 证明 $b\dfrac{\partial z}{\partial x}=a\dfrac{\partial z}{\partial y}$;
        \item 设 $u=(y-z)(z-x)(x-y)$, 证明 $\dfrac{\partial u}{\partial x}+\dfrac{\partial u}{\partial y}+\dfrac{\partial u}{\partial z}=0$;
        \item 设 $z=e^{-(\frac{1}{x}+\frac{1}{y})}$, 证明 $x^2\dfrac{\partial z}{\partial x}+y^2\dfrac{\partial z}{\partial y}=2z$.
    \end{enumerate}

    \item 拉普拉斯方程是指偏微分方程 $\dfrac{\partial^2 u}{\partial x^2}+\dfrac{\partial^2 u}{\partial y^2}=0$. 证明下述函数满足拉普拉斯方程:
    \begin{enumerate}[(1)]\setlength{\itemsep}{5pt}\setlength{\topsep}{15pt}
        \item $u=\ln(x^2+y^2)$;
        \item $u=e^{x}\sin y+e^{y}\cos x$.
    \end{enumerate}

    \item[**7.] 证明
    \[
        f(x, y)=\begin{cases}
            (x^2+y^2)\sin\dfrac{1}{x^2+y^2},&x^2+y^2\not=0,\\
            0,&x^2+y^2=0
        \end{cases}
    \]
    在 $(0, 0)$ 处可微.但偏导数不连续.
    
    \item[*8.] 求下列数的近似值:
    \begin{enumerate}[(1)]\setlength{\itemsep}{5pt}\setlength{\topsep}{15pt}
        \item $(1.04)^{2.02}$;
        \item $\ln(\sqrt[3]{1.03}+\sqrt[4]{0.98}-1)$.
    \end{enumerate} 

    \item[9.] 设有一圆柱形金属工件,高为 $h=10 cm$, 底圆半径 $r=2 cm$, 求高增加 $0.02 cm$, 半径增加 $0.01 cm$ 时, 该工件的体积大致能增加多少?
    
    \item[10.] 求下列函数的全微分:
    \begin{enumerate}[(1)]\setlength{\itemsep}{5pt}\setlength{\topsep}{15pt}
        \item $u=xyz$;
        \item $u=y^{zx}$.
    \end{enumerate}
\end{enumerate}

\section{多元复合函数和隐函数的微分法}

\section{微分法在几何上的应用}

\section{多元函数的极值与最值}

\section{二元函数泰勒公式}

\section{MATLAB求偏导数}