% !TeX program = XeLaTeX
% !TeX root = main.tex
\chapter{多元函数微分学及其应用}\label{cha:8}

% \begin{enumerate}\setlength{\itemsep}{7pt}
% \end{enumerate}

% \begin{enumerate}[(1)]\setlength{\itemsep}{5pt}\setlength{\topsep}{15pt}
% \end{enumerate}

\section{多元函数的概念及其极限和连续}

\begin{enumerate}\setlength{\itemsep}{7pt}
    \item 求下列函数的定义域:
    \begin{enumerate}[(1)]\setlength{\itemsep}{5pt}\setlength{\topsep}{15pt}
        \item $z=\sqrt{\sin(x^2+y^2)}$;
        \item $z=\ln(x^2-3y+2)$;
        \item $z=\sqrt{R^2-x^2-y^2}+\dfrac{1}{\sqrt{x^2+y^2-r^2}}\;\;(R>r>0)$;
        \item $z=\sqrt{x-\sqrt{y}}$;
        \item $z=\dfrac{1}{\sqrt{x+y}}+\dfrac{1}{\sqrt{x-y}}$;
        \item $z=\dfrac{xy}{x^2+y^2}$;
        \item $z=\ln(x-y)-\dfrac{\sqrt{y}}{\sqrt{1-x^2-y^2}}$;
        \item $z=\sqrt{1-x^2}+\sqrt{y^2-4}$.
    \end{enumerate}

    \item 若 $f(x, y)=\dfrac{2xy}{x^2+y^2}$,求 $f\left(1,\dfrac{y}{x}\right)$.
    
    \item 设 $f\left(x+y, \dfrac{y}{x}\right)=x^2-y^2$,求 $f(x, y)$.
    
    \item 设 $z=x+y+f(x-y)$,且当 $y=0$ 时,$z=x^2$,求函数 $f(x)$ 和 $z$ 的表达式.
    
    \item 求下列函数的间断点:
    \begin{enumerate}[(1)]\setlength{\itemsep}{5pt}\setlength{\topsep}{15pt}
        \item $z=\dfrac{x+1}{\sqrt{x^2+y^2}}$;
        \item $z=\dfrac{xy^2}{x+y}$;
        \item $z=\ln(a^2-x^2-y^2)$;
        \item $z=\dfrac{1}{\sin x\cdot\sin y}$.
    \end{enumerate}

    \item 求下列函数的极限:
    \begin{enumerate}[(1)]\setlength{\itemsep}{5pt}\setlength{\topsep}{15pt}
        \item $\displaystyle \lim_{(x, y)\to(0, 1)}\dfrac{\tan(xy)}{x}$;
        \item $\displaystyle \lim_{(x, y)\to(0, 0)}\dfrac{xy}{\sqrt{xy+1}-1}$;
        \item $\displaystyle \lim_{(x, y)\to(0, 2)}\dfrac{\sin(xy)}{x}$;
        \item $\displaystyle \lim_{(x, y)\to(0, 1)}\dfrac{1-xy}{x^2+y^2}$.
    \end{enumerate}

    \item[**7.] 讨论二元函数
    \[
        f(x, y)=\begin{cases}
            \dfrac{xy}{x^2+y^2},&x^2+y^2\not=0,\\
            0,&x^2+y^2=0
        \end{cases}
    \]
    在点 $(0, 0)$ 的连续性.  
\end{enumerate}

\section{偏导数与全微分}

\section{多元复合函数和隐函数的微分法}

\section{微分法在几何上的应用}

\section{多元函数的极值与最值}

\section{二元函数泰勒公式}

\section{MATLAB求偏导数}