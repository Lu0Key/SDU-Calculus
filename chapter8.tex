% !TeX program = XeLaTeX
% !TeX root = main.tex
\chapter{多元函数微分学及其应用}\label{cha:8}

% \begin{enumerate}\setlength{\itemsep}{7pt}
% \end{enumerate}

% \begin{enumerate}[(1)]\setlength{\itemsep}{5pt}\setlength{\topsep}{15pt}
% \end{enumerate}

\section{多元函数的概念及其极限和连续}

\begin{enumerate}\setlength{\itemsep}{7pt}
    \item 求下列函数的定义域:
    \begin{enumerate}[(1)]\setlength{\itemsep}{5pt}\setlength{\topsep}{15pt}
        \item $z=\sqrt{\sin(x^2+y^2)}$;
        \item $z=\ln(x^2-3y+2)$;
        \item $z=\sqrt{R^2-x^2-y^2}+\dfrac{1}{\sqrt{x^2+y^2-r^2}}\;\;(R>r>0)$;
        \item $z=\sqrt{x-\sqrt{y}}$;
        \item $z=\dfrac{1}{\sqrt{x+y}}+\dfrac{1}{\sqrt{x-y}}$;
        \item $z=\dfrac{xy}{x^2+y^2}$;
        \item $z=\ln(x-y)-\dfrac{\sqrt{y}}{\sqrt{1-x^2-y^2}}$;
        \item $z=\sqrt{1-x^2}+\sqrt{y^2-4}$.
    \end{enumerate}

    \item 若 $f(x, y)=\dfrac{2xy}{x^2+y^2}$,求 $f\left(1,\dfrac{y}{x}\right)$.
    
    \item 设 $f\left(x+y, \dfrac{y}{x}\right)=x^2-y^2$,求 $f(x, y)$.
    
    \item 设 $z=x+y+f(x-y)$,且当 $y=0$ 时,$z=x^2$,求函数 $f(x)$ 和 $z$ 的表达式.
    
    \item 求下列函数的间断点:
    \begin{enumerate}[(1)]\setlength{\itemsep}{5pt}\setlength{\topsep}{15pt}
        \item $z=\dfrac{x+1}{\sqrt{x^2+y^2}}$;
        \item $z=\dfrac{xy^2}{x+y}$;
        \item $z=\ln(a^2-x^2-y^2)$;
        \item $z=\dfrac{1}{\sin x\cdot\sin y}$.
    \end{enumerate}

    \item 求下列函数的极限:
    \begin{enumerate}[(1)]\setlength{\itemsep}{5pt}\setlength{\topsep}{15pt}
        \item $\displaystyle \lim_{(x, y)\to(0, 1)}\dfrac{\tan(xy)}{x}$;
        \item $\displaystyle \lim_{(x, y)\to(0, 0)}\dfrac{xy}{\sqrt{xy+1}-1}$;
        \item $\displaystyle \lim_{(x, y)\to(0, 2)}\dfrac{\sin(xy)}{x}$;
        \item $\displaystyle \lim_{(x, y)\to(0, 1)}\dfrac{1-xy}{x^2+y^2}$.
    \end{enumerate}

    \item[**7.] 讨论二元函数
    \[
        f(x, y)=\begin{cases}
            \dfrac{xy}{x^2+y^2},&x^2+y^2\not=0,\\
            0,&x^2+y^2=0
        \end{cases}
    \]
    在点 $(0, 0)$ 的连续性.  
\end{enumerate}

\section{偏导数与全微分}

\begin{enumerate}\setlength{\itemsep}{7pt}
    \item 求下列函数在给定点处的偏导数:
    \begin{enumerate}[(1)]\setlength{\itemsep}{5pt}\setlength{\topsep}{15pt}
        \item $z=x^2+3xy+y^2$,求 $z_x'(1, 2),\;z_y'(1, 2)$;
        \item $z=e^{x^2+y^2}$,求 $z_x'(0, 1),\;z_y'(0, 1)$;
        \item $z=\dfrac{xy(x^2-y^2)}{x^2+y^2}$,求 $z_x'(1, 1),\;z_y'(1, 1)$;
        \item $z=\ln|xy|$,求 $z_x'(-1, -1),\;z_y'(1, 1)$.
    \end{enumerate}

    \item 求下列函数的一阶偏导数:
    \begin{enumerate}[(1)]\setlength{\itemsep}{5pt}\setlength{\topsep}{15pt}
        \item $z=\sin(xy)+\cos^2(xy)$;
        \item $z=\ln(z+\ln y)$;
        \item $z=x^2\arctan\dfrac{y}{x}-y^2\arctan\dfrac{x}{y}$;
        \item $z=x\cdot\ln\dfrac{y}{x}$;
        \item $z=\arcsin\dfrac{x}{y}$;
        \item $z=e^{\frac{x}{y}}+e^{\frac{y}{x}}$;
        \item $z=(1+xy)^{y}$;
        \item $z=\ln\dfrac{\sqrt{x^2+y^2}-x}{\sqrt{x^2+y^2}+x}$
    \end{enumerate}

    \item 求下列函数的二阶偏导数:
    \begin{enumerate}[(1)]\setlength{\itemsep}{5pt}\setlength{\topsep}{15pt}
        \item $z=x\ln(xy)$;
        \item $z=x^4+y^4-4x^2y^2$;
        \item $z=\arctan\dfrac{y}{x}$;
        \item $z=y^x$.
    \end{enumerate}

    \item 求下列函数的全微分:
    \begin{enumerate}[(1)]\setlength{\itemsep}{5pt}\setlength{\topsep}{15pt}
        \item $z=e^{x(x^2+y^2)}$;
        \item $z=\arctan\dfrac{x+y}{x-y}$;
        \item $z=\sqrt{\dfrac{y}{x}}$;
        \item $z=\sqrt{\ln(xy)}$.
    \end{enumerate}

    \item 证明下列各题:
    \begin{enumerate}[(1)]\setlength{\itemsep}{5pt}\setlength{\topsep}{15pt}
        \item 设 $T=2\pi\sqrt{\dfrac{l}{g}}$, 证明 $l\dfrac{\partial T}{\partial l}+g\dfrac{\partial T}{\partial g}=0$;
        \item 设 $z=x^y\cdot y^x$, 证明 $x\dfrac{\partial z}{\partial x}+y\dfrac{\partial z}{\partial y}=z(x+y+\ln z)$;
        \item 设 $z=f(ax+by)$, 证明 $b\dfrac{\partial z}{\partial x}=a\dfrac{\partial z}{\partial y}$;
        \item 设 $u=(y-z)(z-x)(x-y)$, 证明 $\dfrac{\partial u}{\partial x}+\dfrac{\partial u}{\partial y}+\dfrac{\partial u}{\partial z}=0$;
        \item 设 $z=e^{-(\frac{1}{x}+\frac{1}{y})}$, 证明 $x^2\dfrac{\partial z}{\partial x}+y^2\dfrac{\partial z}{\partial y}=2z$.
    \end{enumerate}

    \item 拉普拉斯方程是指偏微分方程 $\dfrac{\partial^2 u}{\partial x^2}+\dfrac{\partial^2 u}{\partial y^2}=0$. 证明下述函数满足拉普拉斯方程:
    \begin{enumerate}[(1)]\setlength{\itemsep}{5pt}\setlength{\topsep}{15pt}
        \item $u=\ln(x^2+y^2)$;
        \item $u=e^{x}\sin y+e^{y}\cos x$.
    \end{enumerate}

    \item[**7.] 证明
    \[
        f(x, y)=\begin{cases}
            (x^2+y^2)\sin\dfrac{1}{x^2+y^2},&x^2+y^2\not=0,\\
            0,&x^2+y^2=0
        \end{cases}
    \]
    在 $(0, 0)$ 处可微.但偏导数不连续.
    
    \item[*8.] 求下列数的近似值:
    \begin{enumerate}[(1)]\setlength{\itemsep}{5pt}\setlength{\topsep}{15pt}
        \item $(1.04)^{2.02}$;
        \item $\ln(\sqrt[3]{1.03}+\sqrt[4]{0.98}-1)$.
    \end{enumerate} 

    \item[9.] 设有一圆柱形金属工件,高为 $h=10 cm$, 底圆半径 $r=2 cm$, 求高增加 $0.02 cm$, 半径增加 $0.01 cm$ 时, 该工件的体积大致能增加多少?
    
    \item[10.] 求下列函数的全微分:
    \begin{enumerate}[(1)]\setlength{\itemsep}{5pt}\setlength{\topsep}{15pt}
        \item $u=xyz$;
        \item $u=y^{zx}$.
    \end{enumerate}
\end{enumerate}

\section{多元复合函数和隐函数的微分法}

\begin{enumerate}\setlength{\itemsep}{7pt}
    \item 求下列函数的导数或偏导数:
    \begin{enumerate}[(1)]\setlength{\itemsep}{5pt}\setlength{\topsep}{15pt}
        \item 设 $z=u\ln v,\;u=x^2,\;v=x^2+y^2$, 求 $\dfrac{\partial z}{\partial x},\;\dfrac{\partial z}{\partial y}$;
        \item 设 $w=ue^{v},\;u=xyz,\;v=x+y+z$, 求 $\dfrac{\partial w}{\partial x},\;\dfrac{\partial w}{\partial y},\;\dfrac{\partial w}{\partial z}$;
        \item 设 $z=e^{x-2y},\;x=\sin t, y=t^3$, 求 $\dfrac{\text{d}z}{\text{d}t}$;
        \item 设 $z=f(e^t, t^2, \sin t)$, $f$ 可微, 求 $\dfrac{\text{d}z}{\text{d}t}$;
        \item 设 $u=f(x, y),\;x=s^2+t^2, y=\cos(s-t)$, 求 $\dfrac{\partial u}{\partial s},\;\dfrac{\partial u}{\partial t}$;
        \item 设 $w=f(x, u, v),\;u=g(x, y),\;v=h(x, y)$, 求 $\dfrac{\partial w}{\partial x},\;\dfrac{\partial w}{\partial y}$;
        \item 设 $u=f\left(x,\dfrac{x}{y}\right)$, 求 $\dfrac{\partial u}{\partial x},\;\dfrac{\partial u}{\partial y}$.
    \end{enumerate}
    
    \item 设 $F(x, y, z)=0$ 且 $F$ 具有连续偏导数, 证明
    \[
        \dfrac{\partial x}{\partial y}\cdot\dfrac{\partial y}{\partial z}\cdot\dfrac{\partial z}{\partial x}=-1.
    \]

    \item 求下列方程所确定的隐函数的导数或偏导数:
    \begin{enumerate}[(1)]\setlength{\itemsep}{5pt}\setlength{\topsep}{15pt}
        \item $y^x=x^y$, 求 $\dfrac{\text{d}y}{\text{d}x}$;
        \item $\sin(xy)=x^2y^2+e^{xy}$, $\dfrac{\text{d}y}{\text{d}x}$;
        \item 设 $e^{z}=xyz$, 求 $\dfrac{\partial z}{\partial x},\;\dfrac{\partial z}{\partial y}$;
        \item 设 $z=e^{xyz}$, 求 $\dfrac{\partial z}{\partial x},\;\dfrac{\partial z}{\partial y}$;
        \item $x+2y+z-2\sqrt{xyz}=0$, 求 $\dfrac{\partial z}{\partial x},\;\dfrac{\partial z}{\partial y}$;
        \item $2xz-2xyz+\ln(xyz)=0$, 求 $\dfrac{\partial z}{\partial x},\;\dfrac{\partial z}{\partial y}$;
        \item $\dfrac{x}{z}=\ln\dfrac{z}{y}$, 求 $\dfrac{\partial z}{\partial x},\;\dfrac{\partial z}{\partial y}$.
    \end{enumerate}

    \item 设 $z=xy+xF(u),\;u=\dfrac{y}{x},\;F(u)$ 为可微函数,证明
    \[
        x\dfrac{\partial z}{\partial x}+y\dfrac{\partial z}{\partial y}=z+xy.
    \]

    \item 求下列函数的二阶偏导数(其中 $f$ 为二阶可微函数):
    \begin{enumerate}[(1)]\setlength{\itemsep}{5pt}\setlength{\topsep}{15pt}
        \item $z=\ln(e^x+e^y)$;
        \item $z=f(xy, y)$;
        \item $z=f(x^2y, xy^2)$.
    \end{enumerate}

    \item 求下列隐函数的二阶偏导数:
    \begin{enumerate}[(1)]\setlength{\itemsep}{5pt}\setlength{\topsep}{15pt}
        \item 设 $x^2+y^2+z^2=4z$, 求 $\dfrac{\partial^2z}{\partial x^2}$;
        \item 设 $z^3-3xyz=a^3$, 求 $\dfrac{\partial^2 z}{\partial x\partial y}$;
        \item[*(3)] 设 $u=f(x, xy, xyz)$, 求 $\dfrac{\partial^2 u}{\partial x^2},\;\dfrac{\partial^2u}{\partial x\partial y}$.
    \end{enumerate}

    \item[**7.] 求由下列方程组所确定的隐函数的导数或偏导数:
    \begin{enumerate}[(1)]\setlength{\itemsep}{5pt}\setlength{\topsep}{15pt}
        \item $\begin{cases}
            z=x^2+y^2,\\
            x^2+2y^2+3z^2=20,
        \end{cases}$
        求 $\dfrac{\text{d}y}{\text{d}x},\;\dfrac{\text{d}z}{\text{d}x}$;
        \item $\begin{cases}
            x+y+u+v=0,\\
            x^2+y^2+u^2+v^2=2,
        \end{cases}$
        求 $\dfrac{\partial u}{\partial x},\;\dfrac{\partial v}{\partial x}$;
        \item $\begin{cases}
            x=e^u+u\sin v,\\
            y=e^u-u\cos v,
        \end{cases}$
        求 $\dfrac{\partial u}{\partial x},\;\dfrac{\partial u}{\partial y},\;\dfrac{\partial v}{\partial x},\;\dfrac{\partial v}{\partial y}$.
    \end{enumerate} 

    \item[**8.] 设函数 $z=f(x, y)$ 在点 $(1, 1)$ 处可微, 且 $f(1, 1)=1,\;\dfrac{\partial f}{\partial x}\bigg|_{(1, 1)}=2,\;\dfrac{\partial f}{\partial y}\bigg|_{(1, 1)}=3,\;\varphi(x)=f(x, f(x,x))$, 求 $\dfrac{\text{d}}{\text{d}x}\varphi^3(x)\bigg|_{x=1}$.
    
    \item[*9.] 设函数 $u=f(x, y, z)$ 有连续偏导数, 且 $z=z(x, y)$ 由方程 $xe^x-ye^y=ze^z$ 所确定, 求 $\text{d}u$.
     
    \item[*10.] 设函数 $z=f(xy, yg(x))$, 其中函数 $f$ 具有二阶连续偏导数, 函数 $g(x)$ 可导且在 $x=1$ 处取得极值 $g(1)=1$. 求 $\dfrac{\partial^2z}{\partial x\partial y}\bigg|_{\substack{x=1\\y=1}}$.  
\end{enumerate}

\section{微分法在几何上的应用}

% \begin{enumerate}\setlength{\itemsep}{7pt}
% \end{enumerate}

\section{多元函数的极值与最值}

% \begin{enumerate}\setlength{\itemsep}{7pt}
% \end{enumerate}

\section{二元函数泰勒公式}

% \begin{enumerate}\setlength{\itemsep}{7pt}
% \end{enumerate}

\section{MATLAB求偏导数}