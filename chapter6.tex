% !TeX program = XeLaTeX
% !TeX root = main.tex
\chapter{无穷级数}\label{cha:6}

\section{常数项级数的概念和性质}


\begin{enumerate}\setlength{\itemsep}{7pt}
    
\item 利用级数收敛的定义判断下列技术的敛散性,如收敛则求其和:  

\begin{enumerate}\setlength{\itemsep}{8pt}
    \item[(1)] $\displaystyle \sum_{n=1}^{\infty}(\sqrt{n+1}-\sqrt{n})$;
    \item[(2)] $\displaystyle \sum_{n=1}^{\infty}\dfrac{1}{(2n-1)(2n+1)}$;
    \item[(3)] $\displaystyle \sum_{n=1}^{\infty}(\sqrt{n+2}-2\sqrt{n+1}+\sqrt{n})$;
    \item[(4)] $\displaystyle \sum_{n=1}^{\infty}\dfrac{1}{(3n-2)(3n+1)}$;
    \item[(5)] $\dfrac{1}{1\cdot2\cdot3}+\dfrac{1}{2\cdot3\cdot4}+\cdots+\dfrac{1}{n\cdot(n+1)\cdot(n+2)}+\cdots$;
    \item[*(6)] $\sin\dfrac{\pi}{6}+\sin\dfrac{2\pi}{6}+\cdots+\sin\dfrac{n\pi}{6}$.
\end{enumerate}  

\item 利用几何级数、调和级数以及收敛级数的性质,判定下列技术的敛散性:
\begin{enumerate}[(1)]\setlength{\itemsep}{10pt}
    \item $\dfrac{1}{3}+\dfrac{1}{6}+\dfrac{1}{9}+\dfrac{1}{12}+\cdots$;
    \item $\dfrac{1}{4}+\dfrac{1}{5}+\dfrac{1}{6}+\dfrac{1}{7}+\cdots$; 
    \item $-\dfrac{8}{9}+\dfrac{8^2}{9^2}-\dfrac{8^3}{9^3}+\cdots$;
    \item $\dfrac{3}{2}+\dfrac{3^2}{2^2}+\dfrac{3^3}{2^3}+\cdots$;
    \item $\left(\dfrac{1}{6}+\dfrac{8}{9}\right)+\left(\dfrac{1}{6^2}+\dfrac{8^2}{9^2}\right)+\left(\dfrac{1}{6^3}+\dfrac{8^3}{9^3}\right)+\cdots$;
    \item $\left(\dfrac{1}{2}+\dfrac{1}{10}\right)+\left(\dfrac{1}{4}+\dfrac{1}{20}\right)+\left(\dfrac{1}{8}+\dfrac{1}{30}\right)+\cdots+\left(\dfrac{1}{2^n}+\dfrac{1}{10n}\right)+\cdots$;
    \item $\dfrac{1}{2}+\dfrac{1}{\sqrt{2}}+\dfrac{1}{\sqrt[3]{2}}+\cdots+\dfrac{1}{\sqrt[n]{2}}+\cdots$;
    \item $\displaystyle\sum_{n=1}^{\infty}\dfrac{1}{\sqrt[n]{n}}$.
\end{enumerate}
\end{enumerate}
    

\section{正项级数的审敛法}

\section{交错级数和任意项级数的审敛法}

\section{幂级数}

\section{函数展开成幂级数}

\section{幂级数的简单应用}

\section{反常积分的审敛法和$\Gamma$函数}

\section{傅里叶级数}

\section{正弦级数、余弦级数和一般区间上的傅里叶级数}

\section{复数形式的傅里叶级数}

\section{用MATLAB计算级数问题}



