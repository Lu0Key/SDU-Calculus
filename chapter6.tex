% !TeX program = XeLaTeX
% !TeX root = main.tex
\chapter{无穷级数}\label{cha:6}

\section{常数项级数的概念和性质}

\begin{enumerate}\setlength{\itemsep}{7pt}
    
\item 利用级数收敛的定义判断下列技术的敛散性,如收敛则求其和:  

\begin{enumerate}\setlength{\itemsep}{8pt}
    \item[(1)] $\displaystyle \sum_{n=1}^{\infty}(\sqrt{n+1}-\sqrt{n})$;
    \item[(2)] $\displaystyle \sum_{n=1}^{\infty}\dfrac{1}{(2n-1)(2n+1)}$;
    \item[(3)] $\displaystyle \sum_{n=1}^{\infty}(\sqrt{n+2}-2\sqrt{n+1}+\sqrt{n})$;
    \item[(4)] $\displaystyle \sum_{n=1}^{\infty}\dfrac{1}{(3n-2)(3n+1)}$;
    \item[(5)] $\dfrac{1}{1\cdot2\cdot3}+\dfrac{1}{2\cdot3\cdot4}+\cdots+\dfrac{1}{n\cdot(n+1)\cdot(n+2)}+\cdots$;
    \item[*(6)] $\sin\dfrac{\pi}{6}+\sin\dfrac{2\pi}{6}+\cdots+\sin\dfrac{n\pi}{6}+\cdots$.
\end{enumerate}  

\item 利用几何级数、调和级数以及收敛级数的性质,判定下列技术的敛散性:
\begin{enumerate}[(1)]\setlength{\itemsep}{10pt}
    \item $\dfrac{1}{3}+\dfrac{1}{6}+\dfrac{1}{9}+\dfrac{1}{12}+\cdots$;
    \item $\dfrac{1}{4}+\dfrac{1}{5}+\dfrac{1}{6}+\dfrac{1}{7}+\cdots$; 
    \item $-\dfrac{8}{9}+\dfrac{8^2}{9^2}-\dfrac{8^3}{9^3}+\cdots$;
    \item $\dfrac{3}{2}+\dfrac{3^2}{2^2}+\dfrac{3^3}{2^3}+\cdots$;
    \item $\left(\dfrac{1}{6}+\dfrac{8}{9}\right)+\left(\dfrac{1}{6^2}+\dfrac{8^2}{9^2}\right)+\left(\dfrac{1}{6^3}+\dfrac{8^3}{9^3}\right)+\cdots$;
    \item $\left(\dfrac{1}{2}+\dfrac{1}{10}\right)+\left(\dfrac{1}{4}+\dfrac{1}{20}\right)+\left(\dfrac{1}{8}+\dfrac{1}{30}\right)+\cdots+\left(\dfrac{1}{2^n}+\dfrac{1}{10n}\right)+\cdots$;
    \item $\dfrac{1}{2}+\dfrac{1}{\sqrt{2}}+\dfrac{1}{\sqrt[3]{2}}+\cdots+\dfrac{1}{\sqrt[n]{2}}+\cdots$;
    \item $\displaystyle\sum_{n=1}^{\infty}\dfrac{1}{\sqrt[n]{n}}$.
\end{enumerate}
\end{enumerate}
    

\section{正项级数的审敛法}

\begin{enumerate}\setlength{\itemsep}{7pt}
    \item 用比较审敛法考察下列级数的敛散性:
    \begin{enumerate}[(1)]\setlength{\itemsep}{10pt}
        \item $1+\dfrac{1}{3}+\dfrac{1}{5}+\dfrac{1}{7}+\cdots$;
        \item $\dfrac{1}{2\cdot5}+\dfrac{1}{3\cdot6}+\cdots+\dfrac{1}{(n+1)(n+4)}+\cdots$;
        \item $1+\dfrac{1+2}{1+2^2}+\dfrac{1+3}{1+3^2}+\cdots$;
        \item $\displaystyle\sum_{n=1}^{\infty}\sin\dfrac{\pi}{2^n}$;
        \item $\displaystyle\sum_{n=1}^{\infty}\dfrac{2+(-1)^n}{4^n}$;
        \item $\displaystyle\sum_{n=1}^{\infty}\dfrac{1}{n\sqrt[n]{n}}$;
        \item $\displaystyle\sum_{n=1}^{\infty}\dfrac{1}{1+a^n}\quad(a>0)$;
        \item $\displaystyle\sum_{n=2}^{\infty}\dfrac{\ln n}{n^{4/3}}$.
    \end{enumerate}

    \item 判定下列级数的敛散性:
    \begin{enumerate}[(1)]\setlength{\itemsep}{10pt}\setlength{\topsep}{15pt}
        \item $\displaystyle\sum_{n=1}^{\infty}\dfrac{1}{\sqrt{n(n+1)}}$;
        \item[*(2)] $\displaystyle \sum_{n=1}^{\infty}(a^{\frac{1}{n}}-1)\quad(a>1)$;
        \item[(3)] $\displaystyle \sum_{n=1}^{\infty}\dfrac{1}{\sqrt{n^3+1}}$;
        \item[(4)] $\displaystyle \sum_{n=1}^{\infty}\dfrac{n^2}{3^n}$;
        \item[(5)] $\displaystyle \sum_{n=1}^{\infty}\dfrac{1}{n}\tan\dfrac{1}{n}$;
        \item[(6)] $\displaystyle \sum_{n=1}^{\infty}\dfrac{(2n-1)!!}{3^n\cdot n!}$;
        \item[(7)] $\displaystyle \sum_{n=1}^{\infty}\dfrac{2^n\cdot n!}{n^n}$;
        \item[(8)] $\displaystyle \sum_{n=1}^{\infty}\dfrac{3^n\cdot n!}{n^n}$;
        \item[(9)] $\displaystyle \sum_{n=1}^{\infty}\dfrac{(n!)^2}{(2n)!}$;
        \item[(10)] $\displaystyle\sum_{n=1}^{\infty}2^n\cdot\sin\dfrac{\pi}{3^n}$;
        \item[(11)] $\displaystyle\sum_{n=1}^{\infty}\dfrac{n^2}{\left(1+\dfrac{1}{n}\right)^n}$;
        \item[*(12)] $\displaystyle\sum_{n=1}^{\infty}\dfrac{1}{(\ln n)^{\ln n}}$;
        \item[(13)] $\displaystyle\sum_{n=1}^{\infty}\sqrt{\dfrac{n+1}{n}}$;
        \item[(14)] $\displaystyle\sum_{n=1}^{\infty}\left(\dfrac{b}{a_n}\right)^n$,其中 $a_n\to a(n\to\infty)$,$a_n$,$b$,$a$ 均为正数;
        \item[(15)] $\displaystyle\sum_{n=1}^{\infty}\dfrac{1}{na+b}\quad(a>0,b>0)$;
        \item[(16)] $\displaystyle\sum_{n=1}^{\infty}\dfrac{4^n}{5^n-3^n}$.        
    \end{enumerate}

    \item 利用级数收敛的必要条件证明:
    \begin{enumerate}[(1)]\setlength{\itemsep}{10pt}\setlength{\topsep}{15pt}
        \item $\displaystyle\lim_{n\to\infty}\dfrac{2^n\cdot n!}{n^n}=0$ ;
        \item $\displaystyle\lim_{n\to\infty}\dfrac{n^n}{(n!)^2}=0$.
    \end{enumerate}

    \item 若 $\displaystyle\lim_{n\to\infty}nu_n=a\not=0$,证明级数 $\displaystyle\sum_{n=1}^{\infty}u_n$ 发散.
    
    \item 设 $\{u_n\}$ 是正项数列,若 $\displaystyle\lim_{n\to\infty}\dfrac{u_{n+1}}{u_n}=l$,证明 $\displaystyle\lim_{n\to\infty}\sqrt[n]{u_n}=l$.
    
    \item 已知 $\displaystyle a_n=\int_0^1x^2(1-x)^n\text{d}x(n=1,2,\cdots)$. 证明 $\displaystyle\sum_{n=1}^{\infty}a_n$ 收敛,并求其和.
    
    \item[*7.] 设 $a_1=2,\;a_{n+1}=\dfrac{1}{2}\left(a_n+\dfrac{1}{a_n}\right)\;(n=1,2,\cdots)$. 证明:
    \begin{enumerate}[(1)]
        \item $\displaystyle\lim_{n\to\infty}a_n$ 存在;
        \item 级数 $\displaystyle\sum_{n=1}^{\infty}\left(\dfrac{a_n}{a_{n+1}}-1\right)$ 收敛.
    \end{enumerate}  

    \item[*8.] 设 $\displaystyle a_{n}=\int_0^{\frac{\pi}{4}}\tan^nx\text{d}x$.
    \begin{enumerate}[(1)]
        \item 求 $\displaystyle\sum_{n=1}^{\infty}\dfrac{1}{n}(a_{n}+a_{n+2})$ 的值;
        \item 试证:对任意的常数 $\lambda>0$,级数 $\displaystyle\sum_{n=1}^{\infty}\dfrac{a_n}{n^{\lambda}}$ 收敛.
    \end{enumerate} 
\end{enumerate}

\section{交错级数和任意项级数的审敛法}

\begin{enumerate}\setlength{\itemsep}{7pt}
    \item 判定下列级数的敛散性:
    \begin{enumerate}[(1)]\setlength{\itemsep}{10pt}\setlength{\topsep}{15pt}
        \item $\displaystyle 1-\dfrac{1}{\sqrt{2}}+\dfrac{1}{\sqrt{3}}-\dfrac{1}{\sqrt{4}}+\cdots$;
        \item $\displaystyle \sum_{n=1}^{\infty}(-1)^{n-1}\dfrac{n}{3^{n-1}}$;
        \item $\displaystyle \sum_{n=1}^{\infty}(-1)^{n-1}\dfrac{1}{\ln(n+1)}$;
        \item $\displaystyle \sum_{n=1}^{\infty}\dfrac{(-1)^n}{n^p}$;
        \item $\displaystyle \sum_{n=1}^{\infty}(-1)^{n}\dfrac{1}{\sqrt[n]{n}}$;
        \item $\displaystyle \sum_{n=1}^{\infty}(-1)^{n}\dfrac{n+2}{n+1}\cdot\dfrac{1}{\sqrt{n}}$.
    \end{enumerate}

    \item 判定下列级数的敛散性,如果收敛,是绝对收敛还是条件收敛?
    \begin{enumerate}[1]\setlength{\itemsep}{10pt}\setlength{\topsep}{15pt}
        \item $\displaystyle \sum_{n=1}^{\infty}\dfrac{1}{n^2}\sin\dfrac{n\pi}{2}$;
        \item $\displaystyle \sum_{n=1}^{\infty}(-1)^n\ln\dfrac{n+1}{n}$;
        \item[*(3)] $\displaystyle \sum_{n=2}^{\infty}\sin\left(n\pi+\dfrac{1}{\ln n}\right)$;
        \item[(4)] $\displaystyle \dfrac{1}{3}\cdot\dfrac{1}{2}-\dfrac{1}{3}\cdot\dfrac{1}{2^2}+\dfrac{1}{3}\cdot\dfrac{1}{2^3}-\dfrac{1}{3}\cdot\dfrac{1}{2^4}+\cdots$;
        \item[(5)] $\displaystyle \sum_{n=1}^{\infty}(-1)^{n+1}\dfrac{2^{n^2}}{n!}$;
        \item[(6)] $\displaystyle \sum_{n=2}^{\infty}\dfrac{(-1)^{n}}{[n+(-1)^n]^p}\quad(p>0)$.  
    \end{enumerate}

    \item 已知 $\displaystyle \sum_{n=1}^{\infty}a_n^2$ 及 $\displaystyle \sum_{n=1}^{\infty}b_n^2$ 收敛,证明级数 $\displaystyle \sum_{n=1}^{\infty}|a_nb_n|,\;\sum_{n=1}^{\infty}(a_n+b_n)^2,\;\sum_{n=1}^{\infty}\dfrac{|a_n|}{n}$ 都收敛.
    
    \item 设 $\displaystyle u_n=\int_{n\pi}^{(n+1)\pi}\dfrac{\sin x}{x}\text{d}x$,证明 $\sum_{n=1}^{\infty}u_n$ 收敛.
    
    \item[*5.] 已知 $f(x)$ 在 $x=0$ 点的某邻域内具有连续的二阶导数,且 $\displaystyle\lim_{x\to0}\dfrac{f(x)}{x}=0$,证明级数 $\displaystyle \sum_{n=1}^{\infty}f\left(\dfrac{1}{n}\right)$ 绝对收敛.  $\bigg($提示:用 $f\left(\dfrac{1}{n}\right)$ 的一阶麦克劳林公式.$\bigg)$
\end{enumerate}

\section{幂级数}

\begin{enumerate}\setlength{\itemsep}{7pt}
    \item 已知函数项级数 $x^2+\dfrac{x^2}{1+x^2}+\dfrac{x^2}{(1+x^2)^2}+\cdots$ 在 $(-\infty, +\infty)$ 上收敛,求其和函数.
    
    \item 求下列幂级数的收敛域:
    \begin{enumerate}[(1)]\setlength{\itemsep}{10pt}\setlength{\topsep}{15pt}
        \item $\displaystyle \sum_{n=1}^{\infty}\dfrac{1}{\sqrt{n}}x^n$;
        \item $\displaystyle \sum_{n=1}^{\infty}\dfrac{2^n}{n^2+1}x^n$;
        \item $\displaystyle \sum_{n=1}^{\infty}(-1)^n\dfrac{x^n}{n}$;
        \item $\displaystyle 1-x+\dfrac{x^2}{2^2}-\dfrac{x^2}{3^2}+\cdots$;
        \item $\displaystyle \dfrac{x}{2}+\dfrac{x^2}{2\cdot4}+\dfrac{x^3}{2\cdot4\cdot6}+\cdots$;
        \item $\displaystyle \sum_{n=1}^{\infty}(-1)^n\dfrac{x^{2n+1}}{2n+1}$;
        \item $\displaystyle \sum_{n=1}^{\infty}\dfrac{(x-5)^n}{\sqrt{n}}$;
        \item $\displaystyle \sum_{n=1}^{\infty}\dfrac{2n-1}{2^n}x^{2n-2}$;
        \item $\displaystyle \sum_{n=1}^{\infty}\left[\dfrac{(-1)^n}{2^n}+3^n\right]x^n$;
        \item 设 $\displaystyle \sum_{n=0}^{\infty}a_nx^n$ 的收敛半径为 $3$,求 $\displaystyle\sum_{n=1}^{\infty}na_n(x-1)^{n+1}$ 的收敛区间.
    \end{enumerate}

    \item[*3.] 求幂级数 $\sum_{n=1}^{\infty}\dfrac{1}{3^n+(-2)^n}\dfrac{x^n}{n}$ 的收敛半径,并讨论该区间端点处的收敛性.
    
    \item[4.] 利用逐项积分或者逐项求导,求下列级数在下列区间内的和函数:
    \begin{enumerate}[(1)]\setlength{\itemsep}{10pt}\setlength{\topsep}{15pt}
        \item $\displaystyle \sum_{n=1}^{\infty}nx^{n-1}\quad(-1<x<1)$;
        \item $\displaystyle \sum_{n=0}^{\infty}\dfrac{x^{2n+1}}{2n+1}\quad(-1<x<1)$,并求 $\displaystyle \sum_{n=0}^{\infty}\dfrac{1}{2n+1}\cdot\dfrac{1}{2^{n+1}}$ 的和;
        \item $\displaystyle \sum_{n=1}^{\infty}\dfrac{2n-1}{2^n}x^{2n-2},\;|x|<\sqrt{2}$,并求 $\displaystyle \sum_{n=1}^{\infty}\dfrac{2n-1}{2^n}$ 的和;
        \item $\displaystyle \sum_{n=1}^{\infty}(2n+1)x^n,\;\;|x|<1$.
    \end{enumerate}   

    \item[*5.] 设 $\displaystyle I_n=\int_0^{\frac{\pi}{4}}\sin^nx\cos x\text{d}x,\;n=0, 1, 2, \cdots$,求 $\displaystyle \sum_{n=0}^{\infty}I_n$.
    
    \item[*6.] 已知 $f_n(x)$ 满足 $f'_n(x)=f_n(x)+x^{n-1}e^x(n$ 为正整数 $)$,且 $f_n(1)=\dfrac{e}{n}$,求函数项级数 $\displaystyle \sum_{n=1}^{\infty}f_n(x)$ 之和.
    
    \item[*7.] 验证函数 $y(x)=1+\dfrac{x^3}{3!}+\dfrac{x^6}{6!}+\cdots+\dfrac{x^{3n}}{(3n)!}+\cdots(-\infty<x<+\infty)$ 满足微分方程
    \[
        y''+y'+y=e^x.
    \]
    并利用以上结果求幂级数 $\displaystyle \sum_{n=0}^{\infty}\dfrac{x^{3n}}{(3n)!}$ 的和函数. 
\end{enumerate}

\section{函数展开成幂级数}

\section{幂级数的简单应用}

\section{反常积分的审敛法和$\Gamma$函数}

\section{傅里叶级数}

\section{正弦级数、余弦级数和一般区间上的傅里叶级数}

\section{复数形式的傅里叶级数}

\section{用MATLAB计算级数问题}



