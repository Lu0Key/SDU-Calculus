% !TeX program = XeLaTeX
% !TeX root = main.tex
\chapter{重积分}\label{cha:9}

% \begin{enumerate}\setlength{\itemsep}{7pt}
% \end{enumerate}

% \begin{enumerate}[(1)]\setlength{\itemsep}{5pt}\setlength{\topsep}{15pt}
% \end{enumerate}


\section{二重积分的概念和性质}

\begin{enumerate}\setlength{\itemsep}{7pt}
    \item 设 $\displaystyle I_1=\iint\limits_{D_1}(x^2+y^2)^3\text{d}\sigma$, 其中 $D_1$ 是矩形闭区域: $-1\leqslant x\leqslant 1,\;-2\leqslant y\leqslant 2$, 又 $I_2=\iint\limits_{D_2}(x^2+y^2)^3\text{d}\sigma$, 其中 $D_2$ 是庆闭区域: $0\leqslant x\leqslant 1,\;0\leqslant y\leqslant 2$, 试用二重积分的几何意义说明 $I_1,\;I_2$ 间的关系.
    
    \item 利用二重积分定义证明:
    \begin{enumerate}[(1)]\setlength{\itemsep}{5pt}\setlength{\topsep}{15pt}
        \item $\displaystyle\iint\limits_{D}kf(x, y)\text{d}\sigma=k\iint\limits_{D}f(x, y)\text{d}\sigma$, 其中 $k$ 为常数;
        \item $\displaystyle\iint\limits_{D}[f(x, y)\pm g(x, y)]\text{d}\sigma=\iint\limits_{D}f(x, y)\text{d}\sigma \pm\iint\limits_{D}g(x, y)\text{d}\sigma$;
        \item $\displaystyle\iint\limits_{D}f(x, y)\text{d}\sigma=\iint\limits_{D_1}f(x, y)\text{d}\sigma+\iint\limits_{D_2}f(x, y)\text{d}\sigma$, 其中 $D=D_1\bigcup D_2,\;D_1,\;D_2$ 是两个无公共内点的闭区域;
        \item $\displaystyle\iint\limits_{D}f(x, y)\text{d}\sigma=\iint\limits_{D}\text{d}\sigma=\sigma$, 其中 $\sigma$ 为区域 $D$ 的面积.
    \end{enumerate}

    \item[*3.] 设函数 $f(x, y)$、$g(x, y)$ 在有界区域 $D$ 上连续, 且 $g(x, y)\geqslant 0$, 试证必存在点 $(\xi, \eta)\in D$, 使
    \[
        \iint\limits_{D}f(x, y)g(x, y)\text{d}\sigma=f(\xi, \eta)\iint\limits_{D}g(x, y)\text{d}\sigma
    \]

    \item[4.] 利用二重积分的性质, 比较下列积分的大小:  
    \begin{enumerate}[(1)]\setlength{\itemsep}{5pt}\setlength{\topsep}{15pt}
        \item $\displaystyle\iint\limits_{D}\sqrt{1+x^2+y^2}\text{d}\sigma$ 与 $\displaystyle\iint\limits_{D}\sqrt{1+x^4+y^4}\text{d}\sigma$, $D$ 为 $x^2+y^2\leqslant 1$;
        \item $\displaystyle\iint\limits_{D}\ln(x+y)\text{d}\sigma$ 与 $\displaystyle\iint[\ln(x+y)]^2\text{d}\sigma$, $D$ 为矩形区域: $3\leqslant x\leqslant 5,\;0\leqslant y\leqslant 1$;
        \item $\displaystyle\iint\limits_{D}(x+y)^2\text{d}\sigma$ 与 $\displaystyle\iint(x+y)^3\text{d}\sigma$, 其中积分区域 $D$ 由圆周 $(x-2)^2+(y-1)^2=2$ 围成.
    \end{enumerate}

    \item[5.] 利用二重积分的性质估计下列积分的值:
    \begin{enumerate}[(1)]\setlength{\itemsep}{5pt}\setlength{\topsep}{15pt}
        \item $\displaystyle I=\iint\limits_{D}(x+y+1)\text{d}\sigma$, $D$ 为 $\{(x, y)\mid 0\leqslant x\leqslant 1,0\leqslant y\leqslant 2\}$;
        \item $\displaystyle I=\iint\limits_{D}(x+xy-x^2-y^2)\text{d}\sigma$, $D$ 为区域 $\{(x, y) \mid 0\leqslant 1, 0\leqslant y \leqslant 2\}$;
        \item $\displaystyle I=\iint\limits_{D}(x^2+4y^2+9)\text{d}\sigma$, $D$ 为圆域 $x^2+y^2\leqslant 4$.
    \end{enumerate} 
\end{enumerate}

\section{二重积分的计算}

\begin{enumerate}\setlength{\itemsep}{7pt}
    \item 将二重积分 $\displaystyle \iint\limits_{D}f(x, y)\text{d}\sigma$ 表示为在直角坐标系下的二次积分(用两种次序给出):
    \begin{enumerate}[(1)]\setlength{\itemsep}{5pt}\setlength{\topsep}{15pt}
        \item $D$ 由 $x=3,\;x=5,\;3x-2y+4=0,\;3x-2y+1=0$ 围成;
        \item $D$ 为 $\{(x, y)\mid x+y\leqslant 1,\;x-y\leqslant 1,\;x\geqslant 0\}$;
        \item $D$ 由 $x+y=2,\;y=x^3,\;y=0$ 围成;
        \item $D$ 为 $\{(x, y)\mid |x|+|y|\leqslant 1\}$.    
    \end{enumerate}

    \item 交换下列二重积分的积分次序:
    \begin{enumerate}[(1)]\setlength{\itemsep}{5pt}\setlength{\topsep}{15pt}
        \item $\displaystyle \int_0^1\text{d}y\int_{y}^{\sqrt{y}}f(x, y)\text{d}x$;
        \item $\displaystyle \int_0^a\text{d}x\int_{x}^{\sqrt{2ax-x^2}}f(x, y)\text{d}y$;
        \item $\displaystyle \int_0^1\text{d}y\int_{-\sqrt{1-y^2}}^{\sqrt{1-y^2}}f(x, y)\text{d}x$;
        \item $\displaystyle \int_0^1\text{d}y\int_0^{2y}f(x, y)\text{d}x+\int_1^3\text{d}y\int_0^{3-y}f(x, y)\text{d}x$.
    \end{enumerate}

    \item 计算下列二重积分:
    \begin{enumerate}[(1)]\setlength{\itemsep}{5pt}\setlength{\topsep}{15pt}
        \item $\displaystyle \iint\limits_{D}\dfrac{x^2}{1+y^2}\text{d}\sigma,\quad D : \{(x, y)\mid 0\leqslant x\leqslant 1,\;0\leqslant y\leqslant 1\}$;
        \item $\displaystyle \iint\limits_{D}x^3y^2\text{d}\sigma,\quad D : \{(x, y)\mid x^2+y^2\leqslant R^2\}$;
        \item $\displaystyle \iint\limits_{D}(x^2+y)\text{d}x\text{d}y,\quad D$ 由曲线 $x^2=y$ 和 $y^2=x$ 围成;
        \item $\displaystyle \iint\limits_{D}\dfrac{x^2}{y^2}\text{d}x\text{d}y,\quad D$ 由 $x=2,\;y=x,\;xy=1$ 围成;
        \item $\displaystyle \iint\limits_{D}\cos(x+y)\text{d}x\text{d}y,\quad D$ 由 $x=0,\;y=\pi,\;x=y$ 围成;
        \item $\displaystyle \iint\limits_{D}\sqrt{1-x^2-y^2}\text{d}x\text{d}y,\quad D$ 是单位圆在第一象限的部分.  
    \end{enumerate}

    \item 给出下列二重积分 $\displaystyle\iint\limits_{D}f(x, y)\text{d}x\text{d}y$ 在极坐标下的二次积分:
    \begin{enumerate}[(1)]\setlength{\itemsep}{5pt}\setlength{\topsep}{15pt}
        \item $D$ 为环域 $1\leqslant x^2+y^2\leqslant 4$;
        \item $D$ 是由 $x^2+y^2-2ax=0,\;y=x$ 围成的第一象限中面积较小的子区域.
    \end{enumerate}

    \item 将下列二重积分化为极坐标形式:
    \begin{enumerate}[(1)]\setlength{\itemsep}{5pt}\setlength{\topsep}{15pt}
        \item $\displaystyle\int_{0}^{R}\text{d}x\int_{0}^{\sqrt{R^2-x^2}}f(x, y)\text{d}y$;
        \item $\displaystyle\int_{0}^{2R}\text{d}y\int_{0}^{\sqrt{2Ry-y^2}}f(x, y)\text{d}x$;
        \item $\displaystyle\int_{0}^{\frac{R}{\sqrt{1+R^2}}}\text{d}x\int_{0}^{Rx}f\left(\dfrac{y}{x}\right)\text{d}y+\int_{\frac{R}{\sqrt{1+R^2}}}^{R}\text{d}x\int_0^{\sqrt{R^2-x^2}}f\left(\dfrac{y}{x}\right)\text{d}y$.
    \end{enumerate}

    \item 利用极坐标系计算下列二重积分:
    \begin{enumerate}[(1)]\setlength{\itemsep}{5pt}\setlength{\topsep}{15pt}
        \item $\displaystyle\int_{0}^{R}\text{d}x\int_{0}^{\sqrt{R^2-x^2}}\ln(1+x^2+y^2)\text{d}y$;
        \item $\displaystyle\iint\limits_{D}\sqrt{R^2-x^2-y^2}\text{d}x\text{d}y,\quad D : x^2+y^2\leqslant Rx$;
        \item $\displaystyle\iint\limits_{D}\arctan\dfrac{y}{x}\text{d}x\text{d}y,\quad D$ 是 $x^2+y^2\leqslant 1$ 在第一象限的部分.
    \end{enumerate}

    \item[**7.] 计算下列曲顶柱体的体积:
    \begin{enumerate}[(1)]\setlength{\itemsep}{5pt}\setlength{\topsep}{15pt}
        \item 曲顶为 $z=1+x^2+y^2$, 区域 $D$ 由 $x=0,\;y=0,\;x=4,\;y=4$ 围成;
        \item 曲顶为 $z=1-\dfrac{x}{a}-\dfrac{y}{b}$, 区域 $D$ 由 $\dfrac{x}{a}+\dfrac{y}{b}=1,\;x=0,\;y=0$ 围成;
        \item 由曲面 $z=\dfrac{h}{R}\sqrt{x^2+y^2}$, 平面 $z=0$ 及圆柱面 $x^2+y^2=R^2$ 围成;
        \item 由坐标面、平面 $x+y=1$ 以旋转抛物面 $z=x^2+y^2$ 围成.    
    \end{enumerate} 

    \item[*8.] 设闭区域 $D : x^2+y^2\leqslant y,\;x\geqslant 0,\;f(x, y)$ 为 $D$ 上的连续函数, 且
    \[
        f(x, y)=\sqrt{1-x^2-y^2}-\dfrac{8}{\pi}\iint\limits_{D}f(u, v)\text{d}u\text{d}v,
    \]
    求 $f(x, y)$.

    \item[*9.] 计算 $\displaystyle \iint\limits_{D}\dfrac{\sqrt{x^2+y^2}}{\sqrt{4a^2-x^2-y^2}}\text{d}x\text{d}y$, 其中 $D$ 是由曲线 $y=-a+\sqrt{a^2-x^2}\;(a>0)$ 和直线 $y=-x$ 围成的区域.
    
    \item[**10.] 计算 $\displaystyle\iint\limits_{D}e^{\max\{x^2, y^2\}}\text{d}x\text{d}y$, 其中 $D=\{(x, y) \mid 0\leqslant x\leqslant 1,\;0\leqslant y\leqslant 1\}$.
    
    \item[11.] 设 $g(x)>0$ 为已知连续函数,在圆域 $D=\{(x, y) \mid x^2+y^2\leqslant a^2(a>0)\}$ 上计算二重积分 $\displaystyle I=\iint\limits_{D}\dfrac{\lambda g(x)+\mu g(y)}{g(x)+g(y)}\text{d}x\text{d}y$, 其中 $\lambda,\;\mu$ 为正常数.
    
    \item[12.] 设 $f(x)$ 为 $[0, 1]$ 上的单调增加的连续函数, 证明:
    \[
        \dfrac{\displaystyle\int_0^1xf^3f(x)\text{d}x}{\displaystyle\int_0^1f^2(x, y)\text{d}x}\geqslant \dfrac{\displaystyle\int_0^1f^3(x)\text{d}x}{\displaystyle\int_0^1f^2(x)\text{d}x}.
    \]

    \item[13.] 设函数 $f(x)$ 在区间 $[0, 1]$ 上连续, 并且 $\displaystyle\int_0^1f(x)\text{d}x=A$, 求 $\displaystyle\int_0^1\text{d}x\int_x^1f(x)f(y)\text{d}y$
\end{enumerate}


\section{三重积分的概念}

\begin{enumerate}\setlength{\itemsep}{7pt}
    \item 设有一物体占据空间区域 $\Omega$, 该物体内任一点 $M(x, y, z)$ 处的体密度为 $\rho=\rho(x, y, z)$, 比热容为 $q=q(x, y, z)$, 且 $\rho(x, y, z),\;q(x, y, z)$ 在 $\Omega$ 上连续, 试求将该物体由温度 $T_1$ 加热到 $T_2$ 所需的热量.
    
    \item 比较积分 $\displaystyle\iiint\limits_{\Omega}(x+y+z)\text{d}V$ 与 $\displaystyle\iiint\limits_{\Omega}(x+y+z)^2\text{d}V$ 值得大小, 其中 $\Omega$ 是由平面 $x+y+z=1$ 与三个坐标面围成的四面体.
    
    \item 估计积分 $\displaystyle\iiint\limits_{\Omega}(x^2+y^2+z^2)\text{d}V$ 的值, 其中
    \[
        \Omega=\{(x, y, z) \mid x^2+y^2+z^2\leqslant R^2\}.
    \]
\end{enumerate}


\section{三重积分的计算}

% \begin{enumerate}\setlength{\itemsep}{7pt}
% \end{enumerate}


\section{重积分的应用}

% \begin{enumerate}\setlength{\itemsep}{7pt}
% \end{enumerate}


\section{用MATLAB计算重积分}

% \begin{enumerate}\setlength{\itemsep}{7pt}
% \end{enumerate}
