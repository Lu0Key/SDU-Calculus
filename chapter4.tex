% !TeX program = XeLaTeX
% !TeX root = main.tex
\chapter{一元函数积分学及其应用}\label{cha:4}

% \begin{enumerate}\setlength{\itemsep}{7pt}
% \end{enumerate}

% \begin{enumerate}[(1)]\setlength{\itemsep}{5pt}\setlength{\topsep}{15pt}
% \end{enumerate}

\section{不定积分}

\begin{enumerate}\setlength{\itemsep}{7pt}
    \item 下列各题中的函数是否是同一函数的原函数:
    \begin{enumerate}[(1)]\setlength{\itemsep}{5pt}\setlength{\topsep}{15pt}
        \item $\ln x,\;\ln 3x$;
        \item $\dfrac{1}{2}\sin^2x,\;-\dfrac{1}{4}\cos 2x$.
    \end{enumerate}

    \item 一曲线通过点 $(e^3,\;3)$,且在任一点处的切线的斜率等于该点横坐标的倒数,
    求该曲线的方程.
    
    \item 求下列简单不定积分:
    \begin{enumerate}[(1)]\setlength{\itemsep}{5pt}\setlength{\topsep}{15pt}
        \item $\displaystyle\int(a-bx^2)^3\text{d}x$;
        \item $\displaystyle\int x\sqrt{x}\text{d}x$;
        \item $\displaystyle\int \dfrac{x^2+1}{\sqrt{x}}\text{d}x$;
        \item $\displaystyle\int (2^x+x^2)\text{d}x$;
        \item $\displaystyle\int \sin^2\dfrac{x}{2}\text{d}x$;
        \item $\displaystyle\int\dfrac{\cos 2x}{\cos^2x\sin^2x}\text{d}x$;
        \item $\displaystyle\int\dfrac{1+\cos^2x}{1+\cos2x}\text{d}x$;
        \item $\displaystyle\int\dfrac{\text{d}x}{x^2(1+x^2)}$;
        \item $\displaystyle\int e^x(1+a^x)\text{d}x$;
        \item $\displaystyle\int \tan^2x\text{d}x$.
    \end{enumerate}

    \item 已知 $f'(x)=\sec^2x+\sin x$,且 $f(0)=1$,求 $f(x)$.
    
    \item 用换元积分法计算下列不定积分:
    \begin{enumerate}[(1)]\setlength{\itemsep}{5pt}\setlength{\topsep}{15pt}
        \item $\displaystyle\int (2x+1)^{100}\text{d}x$;
        \item $\displaystyle\int e^{-5x}\text{d}x$;
        \item $\displaystyle\int \sqrt{4x-1}\text{d}x$;
        \item $\displaystyle\int \sin(a-bx)\text{d}x$;
        \item $\displaystyle\int e^{\cos x}\sin x\text{d}x$;
        \item $\displaystyle\int \dfrac{2x-5}{x^2-5x+7}\text{d}x$;
        \item $\displaystyle\int \dfrac{\text{d}x}{\arcsin^2x\sqrt{1-x^2}}$;
        \item $\displaystyle\int \sin^4x\text{d}x$;
        \item $\displaystyle\int \dfrac{\cos 2x}{1+\sin x\cos x}\text{d}x$;
        \item $\displaystyle\int \dfrac{\text{d}x}{1+e^x}$;
        \item $\displaystyle\int \dfrac{\sin\sqrt{x}}{\sqrt{x}}\text{d}x$;
        \item $\displaystyle\int \dfrac{1+\tan x}{\sin 2x}\text{d}x$;
        \item $\displaystyle\int \dfrac{\tan x}{\sqrt{\cos x}}\text{d}x$;
        \item $\displaystyle\int \dfrac{x^2}{\sqrt{a^2-x^2}}\text{d}x$;
        \item $\displaystyle\int \dfrac{\text{d}x}{(x^2+a^2)^{\frac{3}{2}}}$;
        \item $\displaystyle\int \dfrac{\text{d}x}{x^2\sqrt{x^2+1}}$;
        \item $\displaystyle\int \dfrac{\sqrt{x^2-9}}{x}\text{d}x$;
        \item $\displaystyle\int \dfrac{\text{d}x}{(1+\sqrt[3]{x})\sqrt{x}}$;
        \item $\displaystyle\int \dfrac{\sqrt{x-1}}{x}\text{d}x$;
        \item $\displaystyle\int \dfrac{\text{d}x}{x\sqrt{1-x^4}}$.
    \end{enumerate}

    \item 用分部积分法计算下列不定积分:
    \begin{enumerate}[(1)]\setlength{\itemsep}{5pt}\setlength{\topsep}{15pt}
        \item $\displaystyle\int xe^{-x}\text{d}x$;
        \item $\displaystyle\int \ln x\text{d}x$;
        \item $\displaystyle\int \arcsin x\text{d}x$;
        \item $\displaystyle\int \arctan x\text{d}x$;
        \item $\displaystyle\int x^2\ln x\text{d}x$;
        \item $\displaystyle\int e^{-x}\cos2x\text{d}x$;
        \item $\displaystyle\int \sec^3x\text{d}x$;
        \item $\displaystyle\int x^2\ln(1+x)\text{d}x$.
    \end{enumerate}

    \item[*7.] 计算下列有理函数积分:
    \begin{enumerate}[(1)]\setlength{\itemsep}{5pt}\setlength{\topsep}{15pt}
        \item $\displaystyle\int \dfrac{\text{d}x}{x(x^2+1)}$;
        \item $\displaystyle\int \dfrac{x^2+1}{(x^2-1)(x+1)}\text{d}x$;
        \item $\displaystyle\int \dfrac{1}{x^2+2x+3}\text{d}x$.
    \end{enumerate}

    \item[*8.] 求 $\displaystyle\int |x|\text{d}x$.
    
    \item[**9.] 用合适的方法求下列不定积分:
    \begin{enumerate}[(1)]\setlength{\itemsep}{5pt}\setlength{\topsep}{15pt}
        \item $\displaystyle\int \dfrac{\arcsin\sqrt{x}}{\sqrt{x}}\text{d}x$;
        \item $\displaystyle\int \dfrac{\arctan\sqrt{x}}{\sqrt{x}(1+x)}\text{d}x$;
        \item $\displaystyle\int \dfrac{1}{x^4+1}\text{d}x$;
        \item $\displaystyle\int \dfrac{x-1}{x^2}e^x\text{d}x$.
    \end{enumerate}

    \item[*10.] 已知 $f(x)$ 的一个原函数为 $\ln^2x$,
    求 $\displaystyle\int xf'(x)\text{d}x$.

    \item[*11.] 已知 $\displaystyle\int xf(x)\text{d}x=\arcsin x+C$,
    求 $\displaystyle\int\dfrac{1}{f(x)}\text{d}x$.

    \item[*12.] 设 $f(\ln x)=\dfrac{\ln(1+x)}{x}$,
    计算 $\displaystyle\int f(x)\text{d}x$.
    
    \item[*13.] 设 $f(\sin^2x)=\dfrac{x}{\sin x}$,
    求 $\displaystyle\int\dfrac{\sqrt{x}}{\sqrt{1-x}}f(x)\text{d}x$. 




\end{enumerate}