% !TeX program = XeLaTeX
% !TeX root = main.tex
\chapter{一元函数积分学及其应用}\label{cha:4}

% \begin{enumerate}\setlength{\itemsep}{7pt}
% \end{enumerate}

% \begin{enumerate}[(1)]\setlength{\itemsep}{5pt}\setlength{\topsep}{15pt}
% \end{enumerate}

\section{不定积分}

\begin{enumerate}\setlength{\itemsep}{7pt}
    \item 下列各题中的函数是否是同一函数的原函数:
    \begin{enumerate}[(1)]\setlength{\itemsep}{5pt}\setlength{\topsep}{15pt}
        \item $\ln x,\;\ln 3x$;
        \item $\dfrac{1}{2}\sin^2x,\;-\dfrac{1}{4}\cos 2x$.
    \end{enumerate}

    \item 一曲线通过点 $(e^3,\;3)$,且在任一点处的切线的斜率等于该点横坐标的倒数,
    求该曲线的方程.
    
    \item 求下列简单不定积分:
    \begin{enumerate}[(1)]\setlength{\itemsep}{5pt}\setlength{\topsep}{15pt}
        \item $\displaystyle\int(a-bx^2)^3\text{d}x$;
        \item $\displaystyle\int x\sqrt{x}\text{d}x$;
        \item $\displaystyle\int \dfrac{x^2+1}{\sqrt{x}}\text{d}x$;
        \item $\displaystyle\int (2^x+x^2)\text{d}x$;
        \item $\displaystyle\int \sin^2\dfrac{x}{2}\text{d}x$;
        \item $\displaystyle\int\dfrac{\cos 2x}{\cos^2x\sin^2x}\text{d}x$;
        \item $\displaystyle\int\dfrac{1+\cos^2x}{1+\cos2x}\text{d}x$;
        \item $\displaystyle\int\dfrac{\text{d}x}{x^2(1+x^2)}$;
        \item $\displaystyle\int e^x(1+a^x)\text{d}x$;
        \item $\displaystyle\int \tan^2x\text{d}x$.
    \end{enumerate}

    \item 已知 $f'(x)=\sec^2x+\sin x$,且 $f(0)=1$,求 $f(x)$.
    
    \item 用换元积分法计算下列不定积分:
    \begin{enumerate}[(1)]\setlength{\itemsep}{5pt}\setlength{\topsep}{15pt}
        \item $\displaystyle\int (2x+1)^{100}\text{d}x$;
        \item $\displaystyle\int e^{-5x}\text{d}x$;
        \item $\displaystyle\int \sqrt{4x-1}\text{d}x$;
        \item $\displaystyle\int \sin(a-bx)\text{d}x$;
        \item $\displaystyle\int e^{\cos x}\sin x\text{d}x$;
        \item $\displaystyle\int \dfrac{2x-5}{x^2-5x+7}\text{d}x$;
        \item $\displaystyle\int \dfrac{\text{d}x}{\arcsin^2x\sqrt{1-x^2}}$;
        \item $\displaystyle\int \sin^4x\text{d}x$;
        \item $\displaystyle\int \dfrac{\cos 2x}{1+\sin x\cos x}\text{d}x$;
        \item $\displaystyle\int \dfrac{\text{d}x}{1+e^x}$;
        \item $\displaystyle\int \dfrac{\sin\sqrt{x}}{\sqrt{x}}\text{d}x$;
        \item $\displaystyle\int \dfrac{1+\tan x}{\sin 2x}\text{d}x$;
        \item $\displaystyle\int \dfrac{\tan x}{\sqrt{\cos x}}\text{d}x$;
        \item $\displaystyle\int \dfrac{x^2}{\sqrt{a^2-x^2}}\text{d}x$;
        \item $\displaystyle\int \dfrac{\text{d}x}{(x^2+a^2)^{\frac{3}{2}}}$;
        \item $\displaystyle\int \dfrac{\text{d}x}{x^2\sqrt{x^2+1}}$;
        \item $\displaystyle\int \dfrac{\sqrt{x^2-9}}{x}\text{d}x$;
        \item $\displaystyle\int \dfrac{\text{d}x}{(1+\sqrt[3]{x})\sqrt{x}}$;
        \item $\displaystyle\int \dfrac{\sqrt{x-1}}{x}\text{d}x$;
        \item $\displaystyle\int \dfrac{\text{d}x}{x\sqrt{1-x^4}}$.
    \end{enumerate}

    \item 用分部积分法计算下列不定积分:
    \begin{enumerate}[(1)]\setlength{\itemsep}{5pt}\setlength{\topsep}{15pt}
        \item $\displaystyle\int xe^{-x}\text{d}x$;
        \item $\displaystyle\int \ln x\text{d}x$;
        \item $\displaystyle\int \arcsin x\text{d}x$;
        \item $\displaystyle\int \arctan x\text{d}x$;
        \item $\displaystyle\int x^2\ln x\text{d}x$;
        \item $\displaystyle\int e^{-x}\cos2x\text{d}x$;
        \item $\displaystyle\int \sec^3x\text{d}x$;
        \item $\displaystyle\int x^2\ln(1+x)\text{d}x$.
    \end{enumerate}

    \item[*7.] 计算下列有理函数积分:
    \begin{enumerate}[(1)]\setlength{\itemsep}{5pt}\setlength{\topsep}{15pt}
        \item $\displaystyle\int \dfrac{\text{d}x}{x(x^2+1)}$;
        \item $\displaystyle\int \dfrac{x^2+1}{(x^2-1)(x+1)}\text{d}x$;
        \item $\displaystyle\int \dfrac{1}{x^2+2x+3}\text{d}x$.
    \end{enumerate}

    \item[*8.] 求 $\displaystyle\int |x|\text{d}x$.
    
    \item[**9.] 用合适的方法求下列不定积分:
    \begin{enumerate}[(1)]\setlength{\itemsep}{5pt}\setlength{\topsep}{15pt}
        \item $\displaystyle\int \dfrac{\arcsin\sqrt{x}}{\sqrt{x}}\text{d}x$;
        \item $\displaystyle\int \dfrac{\arctan\sqrt{x}}{\sqrt{x}(1+x)}\text{d}x$;
        \item $\displaystyle\int \dfrac{1}{x^4+1}\text{d}x$;
        \item $\displaystyle\int \dfrac{x-1}{x^2}e^x\text{d}x$.
    \end{enumerate}

    \item[*10.] 已知 $f(x)$ 的一个原函数为 $\ln^2x$,
    求 $\displaystyle\int xf'(x)\text{d}x$.

    \item[*11.] 已知 $\displaystyle\int xf(x)\text{d}x=\arcsin x+C$,
    求 $\displaystyle\int\dfrac{1}{f(x)}\text{d}x$.

    \item[*12.] 设 $f(\ln x)=\dfrac{\ln(1+x)}{x}$,
    计算 $\displaystyle\int f(x)\text{d}x$.
    
    \item[*13.] 设 $f(\sin^2x)=\dfrac{x}{\sin x}$,
    求 $\displaystyle\int\dfrac{\sqrt{x}}{\sqrt{1-x}}f(x)\text{d}x$. 

\end{enumerate}

\section{定积分}

\begin{enumerate}\setlength{\itemsep}{7pt}
    \item 证明定积分性质:
    \[
        \int_a^b\text{d}x=b-a.
    \]

    \item 估计积分值:
    \begin{enumerate}[(1)]\setlength{\itemsep}{5pt}\setlength{\topsep}{15pt}
        \item $\displaystyle\int_1^4(x^2+1)\text{d}x$;
        \item $\displaystyle\int_0^{\frac{\pi}{2}}\sqrt{1+\dfrac{1}{2}\sin^2x}\text{d}x$;
        \item $\displaystyle\int_0^{2}\dfrac{e^{-x}}{x+2}\text{d}x$.
    \end{enumerate}

    \item 不计算积分比较积分值的大小:
    \begin{enumerate}[(1)]\setlength{\itemsep}{5pt}\setlength{\topsep}{15pt}
        \item $\displaystyle\int_0^1x\text{d}x,\quad\int_0^1x^2\text{d}x$;
        \item $\displaystyle\int_0^1e^x\text{d}x,\quad\int_0^1e^{x^2}\text{d}x$;
        \item $\displaystyle\int_0^1x\text{d}x,\quad\int_0^1\ln(1+x)\text{d}x$.
    \end{enumerate}

    \item 求下列各函数的导数:
    \begin{enumerate}[(1)]\setlength{\itemsep}{5pt}\setlength{\topsep}{15pt}
        \item $\displaystyle f(x)=\int_0^xt\sqrt{1+t^2}\text{d}t$;
        \item $\displaystyle f(x)=\int_x^2e^{-t^2}\text{d}t$;
        \item $\displaystyle f(x)=\int_{\sin x}^{\cos x}\cos t^2\text{d}t$;
        \item $\displaystyle \int_{x^2}^{x^3}\dfrac{\text{d}t}{\sqrt{1+t^4}}$.
    \end{enumerate}

    \item 求参数方程 
    $\displaystyle x=\int_0^t\sin^2u\text{d}u,\;y=\int_0^{t^2}\cos\sqrt{u}\text{d}u$ 
    所确定的函数 $y$ 对 $x$ 的一阶导数.

    \item[*6.] 已知函数
    \[
        f(x)=\begin{cases}
            2x,&0\leqslant x\leqslant 1,\\
            2+x,&1<x\leqslant 2,
        \end{cases}
    \] 
    求积分上限函数 $\displaystyle\varphi(x)=\int_0^xf(t)\text{d}t$ 在 $[0,2]$ 上的表达式.

    \item[7.] 计算下列定积分:
    \begin{enumerate}[(1)]\setlength{\itemsep}{5pt}\setlength{\topsep}{15pt}
        \item $\displaystyle \int_{\frac{1}{\sqrt{3}}}^{\sqrt{3}}\dfrac{\text{d}x}{1+x^2}$;
        \item $\displaystyle \int_{-\frac{1}{2}}^{\frac{1}{2}}\dfrac{\text{d}x}{\sqrt{1-x^2}}$;
        \item $\displaystyle \int_{0}^{\frac{\pi}{2}}\sqrt{1-\sin 2x}\text{d}x$;
        \item $\displaystyle \int_1^4\left(\sqrt{x}+\dfrac{1}{\sqrt{x}}\right)\text{d}x$.
    \end{enumerate}

    \item[8.] 求下列极限:
    \begin{enumerate}[(1)]\setlength{\itemsep}{5pt}\setlength{\topsep}{15pt}
        \item $\displaystyle \lim_{x\to0}\dfrac{\displaystyle\int_0^x\sin t^2\text{d}t}{x^3}$;
        \item $\displaystyle \lim_{x\to0^{+}}\dfrac{\displaystyle\int_0^{x^2}t^{\frac{3}{2}}\text{d}t}{\displaystyle\int_0^xt(t-\sin t)\text{d}t}$;
        \item $\displaystyle \lim_{x\to0}\dfrac{\displaystyle\int_0^x\cos(t^2)\text{d}t}{x}$;
        \item[*(4)] $\displaystyle \lim_{x\to0}\dfrac{\displaystyle\int_0^x\left[\int_0^{u^2}\arctan(1+t)\text{d}t\right]\text{d}u}{x(1-\cos x)}$.
    \end{enumerate}

    \item[9.] 设 $f(x)$ 为连续函数,
    $\displaystyle \int_0^xf(x)\text{d}x=x^2(1+x)$,试求 $f(2)$.

    \item[10.] 汽车以 $36 km/h$ 的速度行驶,到某处需要减速停车. 
    设汽车以等加速度 $a=-5m/s^2$ 刹车,问从开始刹车到停车,汽车走了多少距离?
    (提示:先求出开始刹车到停车所用时间.)

    \item[*11.] 设 $f(x)$ 在 $[a,b]$ 上连续,在 $(a,b)$ 内可导且 $f'(x)\leqslant 0$,
    若 $\displaystyle F(x)=\dfrac{1}{x-a}\int_a^xf(t)\text{d}t$,证明在 $(a,b)$ 内有 $F'(x)\leqslant 0$.

    \item[*12.] 设函数 $f(x),\;g(x)$ 在 $[a,b]$ 上连续,且 $g(x)>0$. 证明存在一点 $\xi\in[a,b]$,使
    \[
        \int_a^bf(x)g(x)\text{d}x=f(\xi)\int_a^bg(x)\text{d}x.
    \]
    
    \item[13.] 计算下列定积分:
    \begin{enumerate}[(1)]\setlength{\itemsep}{5pt}\setlength{\topsep}{15pt}
        \item $\displaystyle \int_0^1\dfrac{x\text{d}x}{(x^2+1)^2}$;
        \item $\displaystyle \int_{\frac{1}{\pi}}^{\frac{2}{\pi}}\dfrac{1}{x^2}\sin\dfrac{1}{x}\text{d}x$;
        \item $\displaystyle \int_0^{\ln2}\sqrt{e^x-1}\text{d}x$;
        \item $\displaystyle \int_0^1\dfrac{\sqrt{x}}{1+\sqrt{x}}\text{d}x$;
        \item $\displaystyle \int_0^{\sqrt{2}}\sqrt{2-x^2}\text{d}x$;
        \item $\displaystyle \int_1^{\sqrt{3}}\dfrac{\text{d}x}{x^2\sqrt{1+x^2}}$;
        \item $\displaystyle \int_0^1\dfrac{\arcsin \sqrt{x}}{\sqrt{x(1-x)}}\text{d}x$;
        \item $\displaystyle \int_1^{e^2}\dfrac{\text{d}x}{x\sqrt{1+\ln x}}$;
        \item $\displaystyle \int_{-2}^0\dfrac{\text{d}x}{x^2+2x+2}$;
        \item $\displaystyle \int_0^1te^{-\frac{t^2}{2}}\text{d}t$.
    \end{enumerate}

    \item[14.] 设 $f(x)$ 在区间 $[a,b]$ 上连续,且 $f(x)>0$,
    \[
        F(x)=\int_a^xf(t)\text{d}t+\int_b^x\dfrac{\text{d}t}{f(t)},\;\;x\in[a,b].
    \] 
    证明:
    \begin{enumerate}[(1)]\setlength{\itemsep}{5pt}\setlength{\topsep}{15pt}
        \item $F'(x)\geqslant 2$;
        \item 方程 $F(x)=0$ 在区间 $(a,b)$ 内有且仅有一个根.
    \end{enumerate}

    \item[15.] 设 $f(x)$ 是以 $l$ 为周期的连续函数,证明 $\displaystyle\int_a^{a+l}f(x)\text{d}x$ 的值与 $a$ 无关.
    
    \item[16.] 求下列定积分:
    \begin{enumerate}[(1)]\setlength{\itemsep}{5pt}\setlength{\topsep}{15pt}
        \item $\displaystyle \int_0^1x^2e^x\text{d}x$;
        \item $\displaystyle \int_1^2x\ln\sqrt{x}\text{d}x$;
        \item $\displaystyle \int_0^1x\arctan x\text{d}x$;
        \item $\displaystyle \int_0^{\pi}x\sin 2x\text{d}x$;
        \item[*(5)] $\displaystyle \int_1^e\sin(\ln x)\text{d}x$;
        \item[(6)] $\displaystyle \int_{\frac{1}{e}}^{e}|\ln x|\text{d}x$.
    \end{enumerate}

    \item[*17.] 设 $f(x)=\begin{cases}
        \dfrac{1}{1+e^x},&x<0,\\
        \dfrac{1}{1+x},&x\geqslant0,
    \end{cases}$ 求 $\displaystyle\int_0^2f(x-1)\text{d}x$.

    \item[**18.] 求下列极限:
    \begin{enumerate}[(1)]\setlength{\itemsep}{5pt}\setlength{\topsep}{15pt}
        \item $\displaystyle\lim_{n\to\infty}\dfrac{1}{n}\left(\sqrt{1+\dfrac{1}{n}}+\sqrt{1+\dfrac{2}{n}}+\cdots+\sqrt{1+\dfrac{n}{n}}\right)$;
        \item $\displaystyle\lim_{n\to\infty}\int_0^1\dfrac{x^n}{1+x}\text{d}x$.
    \end{enumerate}

    \item[*19.] 求下列各题:
    \begin{enumerate}[(1)]\setlength{\itemsep}{5pt}\setlength{\topsep}{15pt}
        \item $\displaystyle\dfrac{\text{d}}{\text{d}x}\int_0^x\sin(x-t)^2\text{d}t$;
        \item 设 $f(x)$ 有一个原函数 $\dfrac{\sin x}{x}$,
        求 $\displaystyle\int_{\frac{\pi}{2}}^{\pi}xf'(x)\text{d}x$ 的值.
        \item $\displaystyle \int_0^1\sqrt{2x-x^2}\text{d}x$.
    \end{enumerate}

    \item[*20.] 已知 $f(x)$ 连续,$\displaystyle \int_0^xtf(x-t)\text{d}t=1-\cos x$,
    求 $\displaystyle\int_0^{\frac{\pi}{2}}f(x)\text{d}x$ 的值.

    \item[*21.] 设 $\displaystyle S(x)=\int_0^x|\cos t|\text{d}t$,
    \begin{enumerate}[(1)]\setlength{\itemsep}{5pt}\setlength{\topsep}{15pt}
        \item 当 $n$ 为正整数,且 $n\pi\leqslant x<(n+1)\pi$ 时,证明 $2n\leqslant S(x)< 2(n+1)$;
        \item 求 $\displaystyle\lim_{x\to+\infty}\dfrac{S(x)}{x}$.
    \end{enumerate}

    \item[*22.] 设 $f(x)$ 在区间 $[0,1]$ 上连续,在 $(0,1)$ 内可导,且满足
    \[
        f(1)=3\int_0^{\frac{1}{3}}e^{1-x^2}f(x)\text{d}x.
    \] 
    证明存在 $\xi\in(0,1)$,使得 $f'(\xi)=2\xi f(\xi)$.

    \item[**23.] 已知两曲线 $y=f(x)$ 与 $\displaystyle y=\int_0^{\arctan x}e^{-t^2}\text{d}t$ 
    在点 $(0,0)$ 处的切线相同,写出此切线方程,
    并求极限 $\displaystyle \lim_{n\to\infty} nf\left(\dfrac{2}{n}\right)$.

\end{enumerate}

\section{定积分的应用}

\begin{enumerate}\setlength{\itemsep}{7pt}
    \item 求由下列曲线所围成的图形面积.
    \begin{enumerate}[(1)]\setlength{\itemsep}{5pt}\setlength{\topsep}{15pt}
        \item $y=\dfrac{1}{x}$ 与直线 $y=x$ 及 $x=2$;
        \item $y=x^2,\;4y=x^2$ 及直线 $y=1$;
        \item $y^2=2x$ 和 $y=x-4$;
        \item $x=y^2,\;y=x^2$.
    \end{enumerate}

    \item[*2.] 求由下列各曲线所围成的图形的面积:
    \begin{enumerate}[(1)]\setlength{\itemsep}{5pt}\setlength{\topsep}{15pt}
        \item 星形线 $x=a\cos^3t,\;y=a\sin^3t$;
        \item 圆 $r=2a\cos \theta,\;a>0$.
    \end{enumerate}

    \item[3.] 计算下列各立体体积:
    \begin{enumerate}[(1)]\setlength{\itemsep}{5pt}\setlength{\topsep}{15pt}
        \item $xy=4,\;x=1,\;x=4,\;y=0$,绕 $x$ 轴旋转所得旋转体体积;
        \item $y=x^2,\;x=y^2$,绕 $y$ 轴旋转所得的旋转体体积;
        \item 抛物线 $y^2=4x$ 与直线 $x=1$ 围成的图形绕 $x$ 轴旋转所得的旋转体体积;
        \item[*(4)] 星形线 $x^{\frac{2}{3}}+y^{\frac{2}{3}}=a^{\frac{2}{3}}$ 绕 $x$ 轴旋转所得立体体积;
        \item[(5)] 由 $y=\sqrt{x}$ 与 $x=1,\;x=4$ 及 $x$ 轴所围图形分别绕 $x$ 轴和 $y$ 轴旋转所得两个立体体积.
    \end{enumerate}

    \item[*4.] 计算下列各弧长:
    \begin{enumerate}[(1)]\setlength{\itemsep}{5pt}\setlength{\topsep}{15pt}
        \item 曲线 $y=\ln x$ 相应于 $\sqrt{3}\leqslant x\leqslant \sqrt{8}$ 的一段弧;
        \item 求抛物线 $y=\dfrac{x^2}{2p}$ 由顶点到点 $(\sqrt{2}p,p)$ 的一段弧的长度;
        \item 计算星形线 $x=a\cos^3t,\;y=a\sin^3t$ 的全长;
        \item 对数螺线 $\rho=e^{2\varphi}$ 上 $\varphi=0$ 到 $\varphi+=2\pi$ 的一段弧;
        \item[*(5)] 在摆线 $x=a(t-\sin t),\;y=(1-\cos t)$ 上求分摆线第一拱成 $1:3$ 的点的坐标.
    \end{enumerate}

    \item[5.] 计算下列反常积分的值:
    \begin{enumerate}[(1)]\setlength{\itemsep}{5pt}\setlength{\topsep}{15pt}
        \item $\displaystyle \int_{1}^{+\infty}\dfrac{\text{d}x}{x^4}$;
        \item $\displaystyle \int_0^{+\infty}e^{-\sqrt{x}}\text{d}x$;
        \item $\displaystyle \int_2^{+\infty}\dfrac{\text{d}x}{x^2-1}$;
        \item $\displaystyle \int_0^2\dfrac{\text{d}x}{(1-x)^2}$;
        \item $\displaystyle \int_1^2\dfrac{x}{\sqrt{x-1}}\text{d}x$;
        \item $\displaystyle \int_0^1\dfrac{\text{d}x}{\sqrt{1-x^2}}$;
    \end{enumerate}

    \item[*6.] 已知曲线 $y=a\sqrt{x}(a>0)$ 与曲线 $y=\ln\sqrt{x}$ 
    在点 $(x_0,\;y_0)$ 处有公切线,求
    \begin{enumerate}[(1)]\setlength{\itemsep}{5pt}\setlength{\topsep}{15pt}
        \item 常数 $a$ 的值及切点 $(x_0,y_0)$;
        \item 两曲线与 $x$ 轴围成的在 $x$ 轴上方的图形面积.
    \end{enumerate}

    \item[*7.] 半径等于 $r m$ 的半球形水池,其中充满了谁,把池内的水完全吸尽,
    需作多少功?(水的密度 $\rho=1\;000kg/m^3$,设重力加速度为 $g$)

    \item[*8.] 由实验知道,弹簧在拉伸过程中,需要的力 $F$(单位:$N$)与伸长量 $s$(单位:$cm$)成正比,
    即 $F=ks$($k$ 是劲度系数). 如果把弹簧由原长拉伸 $6cm$,计算所作的功.

    \item[*9.] 一底为 $8\;cm$,高为 $6\;cm$ 的等腰三角形片,铅值地沉没在水中,
    顶在上,底在下且与水面平行,而顶离水面 $3\;cm$,试求它每面所受的压力(设重力加速度为 $g$).

    \item[**10.] 设有一长度为 $l$,线密度为 $\rho$ 的均匀细长棒,
    在棒的一端垂直距离为 $a$ 单位处有一质量为 $m$ 的质点 $M$,
    试求这细棒对指点 $M$ 的引力.

    \item[*11.] 设抛物线 $y=ax^2+bx+2\ln c$ 过原点,当 $0\leqslant x\leqslant 1$ 时,
    $y\geqslant 0$,又已知该抛物线与 $x$ 轴及直线 $x=1$ 所围图形的面积为 $\dfrac{1}{3}$,
    试确定 $a,\;b,\;c$,使此图形绕 $x$ 轴旋转一周而成的旋转体的体积 $V$ 最小.

    \item[*12.] 某闸门的形状与大小如图所示(还没画),其中直线 $l$ 为对称轴,
    闸门的上部为矩形 $ABCD$,下部由二次抛物线与线段 $AB$ 所围成. 当水面与闸门的上端相平时,
    欲使闸门矩阵部分承受的水压力与闸门下部承受的压力之比为 $5:4$,闸门矩阵部分的高 $h$ 应为多少米?

    \item[*13.] 设 $D_1$ 是由抛物线 $y=2x^2$ 和直线 $x=a\;(0<a<2),\;x=2$ 及 $y=0$ 所围成的平面区域;
    $D_2$ 是由抛物线 $y=2x^2$ 和直线 $y=0,\;x=a$ 所围成的平面区域.
    \begin{enumerate}[(1)]\setlength{\itemsep}{5pt}\setlength{\topsep}{15pt}
        \item 求 $D_1$ 绕 $x$ 轴旋转而成的旋转体体积 $V_1,\;D_2$ 绕 $y$ 轴旋转而成的旋转体体积 $V_2$;
        \item 当 $a$ 为何值时,$V_1+V_2$ 取得最大值?并求此最大值.
    \end{enumerate}   

\end{enumerate}