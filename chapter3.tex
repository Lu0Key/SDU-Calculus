% !TeX program = XeLaTeX
% !TeX root = main.tex
\chapter{中值定理和导数的应用}\label{cha:3}

% \begin{enumerate}\setlength{\itemsep}{7pt}
% \end{enumerate}

% \begin{enumerate}[(1)]\setlength{\itemsep}{5pt}\setlength{\topsep}{15pt}
% \end{enumerate}

\section{微分中值定理}

\begin{enumerate}\setlength{\itemsep}{7pt}
    \item 下列函数在所给区间上是否满足罗尔定理的条件?若满足,求出定理结论中的内点 $\xi$:
    \begin{enumerate}[(1)]\setlength{\itemsep}{5pt}\setlength{\topsep}{15pt}
        \item $f(x)=x^2-5x+6,\quad[2,3]$;
        \item $f(x)=\sin x,\quad[0,\pi]$;
        \item $f(x)=\dfrac{1}{1+x^2},\quad[-2,2]$;
        \item $f(x)=1-\sqrt[3]{x^2},\quad[-1,1]$.
    \end{enumerate}

    \item 下列函数在所给区间上是否满足拉格朗日中值定理的条件?若满足,求出定理结论中的内点 $\xi$:
    \begin{enumerate}[(1)]\setlength{\itemsep}{5pt}\setlength{\topsep}{15pt}
        \item $f(x)=4x^3-5x^2+x-2,\quad[0,1]$;
        \item $f(x)=e^x,\quad[0,\ln2]$;
        \item $f(x)=\ln x,\quad[1,2]$.
    \end{enumerate}

    \item 证明恒等式
    \begin{enumerate}[(1)]\setlength{\itemsep}{5pt}\setlength{\topsep}{15pt}
        \item $\arcsin x+\arccos x=\dfrac{\pi}{2},\;\;-1\leqslant x\leqslant 1$;
        \item $\arctan x+\arctan\dfrac{1}{x}=\dfrac{\pi}{2},\;\;x>0$.
    \end{enumerate}

    \item 应用拉格朗日中值公式证明不等式:
    \begin{enumerate}[(1)]\setlength{\itemsep}{5pt}\setlength{\topsep}{15pt}
        \item $|\sin x_2-\sin x_1|\leqslant |x_2-x_1|$;
        \item $|\arctan x_2-\arcsin x_1|\leqslant |x_2-x_1|$;
        \item $e^x>e\cdot x,\;\;x>1$;
        \item $\dfrac{a-b}{a}<\ln\dfrac{a}{b}<\dfrac{a-b}{b}$.
    \end{enumerate}
    
    \item 对函数 $f(x)=\sin x$ 和 $g(x)=x+\cos x$ 在区间 $\left[0,\;\dfrac{\pi}{2}\right]$ 上验证柯西公式
    \[
        \dfrac{f(b)-f(a)}{g(b)-g(a)}=\dfrac{f'(\xi)}{g'(\xi)},\quad a<\xi<b.
    \]

    \item 若方程 $a_0x^n+a_1x^{n-1}+\cdots+a_{n-1}x=0$ 有一个正根 $x_0$,
    证明方程 $a_0nx^{n-1}+a_1(n-1)x^{n-2}+\cdots+a_{n-1}=0$ 必有一个小于 $x_0$ 的正根.

    \item 若函数 $f(x)$ 在 $(a,b)$ 内具有二阶导数且 $f(x_1)=f(x_2)=f(x_3)$,
    其中 $a<x_1<x_2<x_3<b$,证明:在 $(x_1,x_3)$ 内至少有一点 $\xi$,使 $f''(\xi)=0$.

    \item 设函数 $f(x)$ 在 $[0,3]$ 上连续,在 $(0,3)$ 内可导,且 $f(0)+f(1)+f(2)=3,\;f(3)=1$. 
    试证:必存在 $\xi\in(0,3)$,使得 $f'(\xi)=0$.

    \item 证明:若函数 $f(x)$ 在 $(-\infty,+\infty)$ 内满足关系式 $f'(x)=f(x)$,
    且 $f(0)=1$,则 $f(x)=e^x$.

    \item 设函数 $y=f(x)$在$x=0$的某个领域内具有$n$ 阶导数,且 $f(0)=f'(0)=\cdots=f^{(n-1)}(0)=0$,
    试用柯西中值定理证明
    \[
        \dfrac{f(x)}{x^n}=\dfrac{f^{(n)}(\theta x)}{n!},\;\;0<\theta<1.
    \]

    \item 设函数 $f(x),\;g(x)$ 在 $[a,b]$ 上连续,
    在 $(a,b)$ 内具有二阶导数且存在相等的最大值,$f(a)=g(a),\;f(b)=g(b)$. 
    证明:
    \begin{enumerate}[(1)]\setlength{\itemsep}{5pt}\setlength{\topsep}{15pt}
        \item 存在 $\eta\in(a,b)$,使得 $f(\eta)=g(\eta)$;
        \item 存在 $\xi\in(a,b)$,使得 $f''(\xi)=g''(\xi)$.
    \end{enumerate}
    
    \item 设 $x_1x_2>0$,证明:$x_1e^{x_2}-x_2e^{x_1}=(1-\xi)e^{\xi}(x_1-x_2)$,
    其中 $\xi$ 在 $x_1$ 与 $x_2$ 之间.

    \item 设 $f(x)$ 在 $[a,b]$ 上连续,在 $(a,b)$ 内可导,且 $f(a)=f(b)=1$,
    试证:存在 $\xi,\;\eta\in(a,b)$,使 $e^{\eta-\xi}[f(\eta)+f'(\eta)]=1$.
\end{enumerate}

\section{洛必达法则}

\begin{enumerate}\setlength{\itemsep}{7pt}
    \item 求下列极限:
    \begin{enumerate}[(1)]\setlength{\itemsep}{5pt}\setlength{\topsep}{15pt}
        \item $\displaystyle\lim_{x\to0}\dfrac{x-\sin x}{x^3}$;
        \item $\displaystyle\lim_{x\to0}\dfrac{e^{x}-e^{-x}}{x}$;
        \item $\displaystyle\lim_{x\to0}\dfrac{e^x-1}{x^2-x}$;
        \item $\displaystyle\lim_{x\to\frac{\pi}{2}}\dfrac{\ln \sin x}{(\pi-2x)^2}$;
        \item $\displaystyle\lim_{x\to\pi}\dfrac{\sin 3x}{\tan 5x}$;
        \item $\displaystyle\lim_{x\to a}\dfrac{x^m-a^m}{x^n-a^n}$;
        \item $\displaystyle\lim_{x\to\frac{\pi}{2}^{-}}\dfrac{\ln\cot x}{\tan x}$;
        \item $\displaystyle\lim_{x\to+\infty}\dfrac{\ln\left(1+\dfrac{1}{x}\right)}{\text{arccot}\; x}$.
    \end{enumerate}

    \item 求下列极限:
    \begin{enumerate}[(1)]\setlength{\itemsep}{5pt}\setlength{\topsep}{15pt}
        \item $\displaystyle\lim_{x\to0^+}xe^{\frac{1}{x}}$;
        \item $\displaystyle\lim_{x\to0^+}x^2\ln x$;
        \item $\displaystyle\lim_{x\to1}\left(\dfrac{2}{x^2-1}-\dfrac{1}{x-1}\right)$;
        \item $\displaystyle\lim_{x\to\frac{\pi}{2}}(\sec x-\tan x)$;
        \item $\displaystyle\lim_{x\to1}\left(\dfrac{x}{x-1}-\dfrac{1}{\ln x}\right)$;
        \item $\displaystyle\lim_{x\to1}x^{\frac{1}{1-x}}$;
        \item $\displaystyle\lim_{x\to0}\left(\dfrac{2}{\pi}\arccos x\right)^{\frac{1}{x}}$;
        \item $\displaystyle\lim_{x\to+\infty}(1+x)^{\frac{1}{\sqrt{x}}}$;
        \item $\displaystyle\lim_{x\to0}(\sin x)^x$;
        \item $\displaystyle\lim_{x\to0}\left(\dfrac{3-e^x}{2+x}\right)^{\dfrac{1}{\sin x}}$.
    \end{enumerate}

    \item 求下列极限:
    \begin{enumerate}[(1)]\setlength{\itemsep}{5pt}\setlength{\topsep}{15pt}
        \item $\displaystyle\lim_{x\to0}\dfrac{1}{x}\left(\dfrac{1}{\sin x}-\dfrac{1}{\tan x}\right)$;
        \item $\displaystyle\lim_{x\to0}\left(\dfrac{1}{x}+\dfrac{1}{x^2}\ln(1-x)\right)$;
        \item $\displaystyle\lim_{x\to0}\dfrac{\ln(1+x^2)}{\sec x-\cos x}$;
        \item $\displaystyle\lim_{x\to+\infty}\left(\dfrac{\pi}{2}-\arctan x\right)^{\frac{1}{x}}$;
        \item $\displaystyle\lim_{x\to+\infty}\dfrac{\ln(x\ln x)}{x^{\alpha}},\;\alpha>0$;
        \item $\displaystyle\lim_{n\to\infty}n^3(a^{\frac{1}{n}}-a^{\sin\frac{1}{n}}),\;a>0$.
    \end{enumerate}

    \item 验证极限 $\displaystyle\lim_{x\to+\infty}\dfrac{e^x-e^{-x}}{e^x+e^{-x}}$ 
    及  $\displaystyle\lim_{x\to0}\dfrac{x^2\sin\dfrac{1}{x}}{\sin x}$ 存在,但不能用洛必达法则求出.

    \item 讨论函数
    \[
        f(x)=\begin{cases}
            \left[\dfrac{(1+x)^{\frac{1}{x}}}{e}\right]^{\frac{1}{x}},&x>0,\\
            e^{-\frac{1}{2}},&x\leqslant0
        \end{cases}
    \]
    在 $x=0$ 的连续性.

    \item 试确定常数 $A,\;B,\;C$ 的值,使得 $e^x(1+Bx+Cx^2)=1+Ax+o(x^3)$,
    其中 $o(x^3)$ 是当 $x\to0$ 时比 $x^3$ 高阶的无穷小.

    \item 设 $f''(x)$ 存在,
    求证 $\displaystyle\lim_{h\to0}\dfrac{f(x+2h-2f(x+h)+f(x))}{h^2}=f''(x)$.

    \item 设函数 $f(x)$ 在 $x=0$ 的某领域内具有一阶连续导数,且 $f(0)\not=0,\;f'(0)\not=0$,
    若 $af(h)+bf(2h)-f(0)$ 在 $h\to0$ 时是比 $h$ 高阶的无穷小,试确定 $a,\;b$ 的值.

    \item 设
    \[
        f(x)=\begin{cases}
            \dfrac{g(x)-e^{-x}}{x},&x\not=0,\\
            0,&x=0,
        \end{cases}
    \]
    其中 $g(x)$ 有二阶连续导数,且 $g(0)=1,\;g'(0)=-1$,
    \begin{enumerate}[(1)]\setlength{\itemsep}{5pt}\setlength{\topsep}{15pt}
        \item 求 $f'(x)$;
        \item 讨论 $f'(x)$ 在 $(-\infty,+\infty)$ 上的连续性.
    \end{enumerate}


\end{enumerate}

\section{泰勒中值定理}

\begin{enumerate}\setlength{\itemsep}{7pt}
    \item 按 $(x+1)$ 的乘幂展开多项式 $x^3+3x^2-2x+4$.
    
    \item 求 $f(x)=\dfrac{1}{x}$ 在点 $x_0=-1$ 的 $n$ 阶泰勒公式.
    
    \item 求 $f(x)=\sqrt{x}$ 在点 $x_0=4$ 处的三阶泰勒公式.
    
    \item 求 $f(x)=xe^x$ 的 $n$ 阶麦克劳林公式(佩亚诺型余项).
    
    \item 求 $f(x)=\tan x$ 的二阶麦克劳林公式.
    
    \item 利用麦克劳林公式求极限:
    \begin{enumerate}[(1)]\setlength{\itemsep}{5pt}\setlength{\topsep}{15pt}
        \item $\displaystyle\lim_{x\to0}\dfrac{\ln(1+x)-\left(x-\dfrac{5}{2}x^2\right)}{x^2}$;
        \item $\displaystyle\lim_{x\to0}\dfrac{x^2(\sqrt{1+x\sin x}+\sqrt{\cos x})}{1+x\sin x-\cos x}$;
        \item $\displaystyle\lim_{x\to0}\dfrac{x\ln(1+x)}{e^x-x-1}$;
        \item $\displaystyle\lim_{x\to0}\left(\dfrac{1}{x}-\dfrac{1}{\sin x}\right)\dfrac{1}{x}$.
    \end{enumerate}

    \item 应用三阶泰勒公式求下列各数的近似值,并估计误差.
    \begin{enumerate}[(1)]\setlength{\itemsep}{5pt}\setlength{\topsep}{15pt}
        \item $\sqrt[3]{30}$;
        \item $\sqrt{e}$.
    \end{enumerate}

    \item 设当 $x\to0$ 时,$\sin x-(ax^3+bx^2+cx)$ 是比 $x^3$ 高阶的无穷小,求 $a,\;b,\;c$ 的值.
    
    \item 设 $f(x)$ 在点 $x=0$ 的某个领域内二阶可导,
    且 $\displaystyle\lim_{x\to0}\dfrac{\sin x+xf(x)}{x^3}=\dfrac{1}{2}$,
    求 $f(0),\;f'(0)$ 及 $f''(0)$ 的值.

    \item 求函数 $f(x)=x^2\ln(1+x)$ 在 $x=0$ 处的 $n$ 阶导数 $f^{(n)}(0)\;(n\geqslant3)$.
    
    \item 设 $\displaystyle\lim_{x\to0}\dfrac{f(x)}{x}=1$,且 $f''(x)>0$,证明 $f(x)\geqslant x$.
    
    \item 设 $f(x)$ 在闭区间 $[-1,\;1]$ 上具有三阶连续导数,且 $f(-1)=0,\;f(1)=1,\;f'(0)=0$. 
    证明:存在 $\xi\in(-1,\;1)$,使 $f'''(\xi)=3$.


\end{enumerate}

\section{函数的单调性、极值和最大最小值}

\begin{enumerate}\setlength{\itemsep}{7pt}
    \item 求下列函数的极值:
    \begin{enumerate}[(1)]\setlength{\itemsep}{5pt}\setlength{\topsep}{15pt}
        \item $y=x^3-x^2+1$;
        \item $y=-x^4+2x^2$;
        \item $y=\sqrt{x}e^x$;
        \item $y=\dfrac{x^2}{1+x}$;
        \item $y=x-\ln (1+x)$;
        \item[*(6)] 函数 $y=f(x)$ 由方程 $x^3+y^3-3xy=0(x>0)$ 所确定,求其极值点;
        \item[(7)] $y=\dfrac{3x^2+4x+4}{x^2+x+1}$;
        \item[(8)] $y=2e^x+e^{-x}$;
        \item[(9)] $y=x+\sqrt{1-x}$;
        \item[*(10)] $\begin{cases}
            x=\dfrac{1}{2}(t+1)^2,\\
            \\
            y=\dfrac{1}{2}(t-1)^2;
        \end{cases}\quad(t\geqslant0)$
        \item[(11)] $y=2-(x-1)^{\frac{2}{3}}$;
        \item[(12)] $y=x+\tan x$. 
    \end{enumerate}

    \item 求下列函数在给定区间上的最大、最小值:
    \begin{enumerate}[(1)]\setlength{\itemsep}{5pt}\setlength{\topsep}{15pt}
        \item $y=x^4-8x^2+2,\qquad-1\leqslant x\leqslant 3$;
        \item $y=|x^2-3x+2|,\qquad-10\leqslant x\leqslant10$;
        \item $y=x-\sin x,\qquad\qquad-\pi\leqslant x\leqslant \pi$;
        \item $y=x+\sqrt{1-x},\quad\qquad -5\leqslant x\leqslant 1$;
        \item $y=x^{\frac{2}{3}}-(x^2-1)^{\frac{1}{3}},\;\quad -2\leqslant x\leqslant 2$.
    \end{enumerate}

    \item[*3.] 利用单调性、极值(最值)证明下列不等式:
    \begin{enumerate}[(1)]\setlength{\itemsep}{5pt}\setlength{\topsep}{15pt}
        \item 当 $0<x<\dfrac{\pi}{2}$ 时,$\dfrac{\sin x}{x}>\dfrac{2}{\pi}$;
        \item 设 $p$ 是大于 $1$ 的正整数,$\dfrac{1}{p}+\dfrac{1}{q}=1$,证明:对任意正数 $x$,
        \[
            \dfrac{1}{p}x^p+\dfrac{1}{q}\geqslant x.
        \]
        \item[*(3)] 设 $0<a<b$,证明不等式
        \[
            \dfrac{2a}{a^2+b^2}<\dfrac{\ln b-\ln a}{b-a}<\dfrac{1}{\sqrt{ab}}.
        \]
        \item[(4)] 设 $e<a<b<e^2$,证明:$\ln^2b-\ln^2a>\dfrac{4}{e^2}(b-a)$. 
    \end{enumerate}

    \item[4.] 证明函数 $f(x)=\left(1+\dfrac{1}{x}\right)^x$ 在区间 $(0,\;+\infty)$ 上单调增加.
    
    \item[5.] 确定二次三项式 $y=x^2+px+q$ 的系数 $p$ 与 $q$,使二次三项式当 $x=1$ 时有最小值 $y=3$.
    
    \item[6.] 将一长为 $l$ 的铁丝截为两段并将其中一段围成正方形,另一段围成圆形,
    为使正方形与圆形面积之和最小,问这两段铁丝的长应各为多少?  

    \item[7.] 设 $a>1$,$f(t)=a^t-at$ 在 $(-\infty,\;+\infty)$ 内的驻点为 $t(a)$. 
    问 $a$ 为何值时,其体积 $V$ 最小,并求出最小值.

    \item[8.] 作半径为 $r$ 的球的外切正圆锥,问此圆锥的高h为何值时,其体积 $V$ 最小,并求出最小值.
    
    \item[9.] 从一块半径为 $R$ 的圆铁片上剪去一个扇形后,再做成一个漏斗. 
    问留下的扇形的圆心角 $\varphi$ 为多大时,做成的漏斗容积最大?

    \item[10.] 某商品进价为 $a$ (元/件),根据以往经验,当销售价为 $b$(元/件)时,
    销售量为 $c$ 件($a,b,c$ 均为正数,且 $b\geqslant \dfrac{4}{3}a$),市场调查表明,
    销售价每下降 $10\%$,销售量可增加 $40\%$. 现决定一次性降价,
    试问:当销售价为多少时,可获得最大利润?并求出最大利润.
      

\end{enumerate}

\section{曲线的凹凸性和函数作图}

\begin{enumerate}\setlength{\itemsep}{7pt}
    \item 求下列函数的凹凸性及拐点:
    \begin{enumerate}[(1)]\setlength{\itemsep}{5pt}\setlength{\topsep}{15pt}
        \item $y=x^3-6x^2+12x+4$;
        \item $y=\dfrac{1}{x+1}$;
        \item $y=(1+x^2)e^x$;
        \item $y=3x^5-5x^4$;
        \item $y=(x-2)^{\frac{5}{3}}-\dfrac{5}{9}x^2$;
        \item $y=x+\dfrac{x}{x^2-1}$.
    \end{enumerate}

    \item 做出下列函数的图形(多半不会搞):
    \begin{enumerate}[(1)]\setlength{\itemsep}{5pt}\setlength{\topsep}{15pt}
        \item $y=x-\dfrac{3}{2}x^{\frac{2}{3}}$;
        \item $y=\dfrac{x}{1+x^2}$;
        \item $y=e^{-(x-1)^2}$;
        \item $y=\ln(x^2+1)$;
        \item $y=\dfrac{\cos x}{\cos 2x}$.
        \item $y=(x+6)e^{\frac{1}{x}}$.
    \end{enumerate}

    \item 求曲线 $y=(2x-1)e^{\frac{1}{x}}$ 的斜渐近线方程.
    
    \item 判断曲线 $y=xe^{\frac{1}{x^2}}$ 有无水平渐近线、铅直渐近线、斜渐近线,
    有的话,请求出它们.


\end{enumerate}

\section{弧微分$\quad$曲率}

\begin{enumerate}\setlength{\itemsep}{7pt}
    \item 求下列曲线在指定点的曲率和曲率半径.
    \begin{enumerate}[(1)]\setlength{\itemsep}{5pt}\setlength{\topsep}{15pt}
        \item $4x^2+y^2=4$ 在点 $(0,2)$ 处;
        \item $y=\ln\sin x$ 在点 $\left(\dfrac{\pi}{2},\;0\right)$ 处;
        \item $\begin{cases}
            x=\cos^3t,\\
            y=\sin^3t
        \end{cases}$ 在 $t=\dfrac{\pi}{4}$ 处;
        \item $r=a(1+\cos \theta)$ 在 $\theta=0$ 处.
    \end{enumerate}

    \item 求 $y=e^x$ 曲率最大的点. 
    
    \item 求 $y=x^2$ 上曲率半径最小的点.
    
    \item 设抛物线 $y=ax^2+bx+c$ 在 $x=0$ 处与曲线 $y=e^x$ 相切,并有共同的曲率半径,求 $a,\;b,\;c$.
    
    \item 设摆线 $\begin{cases}
        x=a(t-\sin t),\\
        y=a(1-\cos t),
    \end{cases},\;a>0,\;t\in(0,2\pi)$. 问 $t$ 为何值时,曲率最小,并求出最小曲率和该点处的曲率半径.

    \item 设 $R=R(x)$ 是抛物线  $y=\sqrt{x}$ 上任一点 $M(x, y)\;(x\geqslant 1)$ 处的曲率半径,$s=s(x)$ 是抛物线上介于点 $A(1,1)$ 与 $M$ 之间的弧长,计算
    \[
        3R\dfrac{\text{d}^2R}{\text{d}s^2}-\left(\dfrac{\text{d}R}{\text{d}s}\right)^2
    \]
    的值.

    \item 求曲线 $y=\ln x$ 在与 $x$ 轴交点处的曲率圆方程.
    
    \item 求曲线 $y=\tan x$ 在点 $\left(\dfrac{\pi}{4},\;1\right)$ 处的曲率圆方程.

\end{enumerate}